% Dependencies:
%	*	mathrsfs	- for mathscr font
%	*	amsfonts	- for mathfrak and mathbb (upper case) letters
%	*	amssym		- for backepsilon
%	*	dsfont		- for mathds
\newglossary*{notation}{Notation}

% ==============================================================================
% Code
% ==============================================================================
	\newcommand{\code}[1]{\ensuremath{\texttt{#1}}}
	\newglossaryentry{not:code}
	{%
		name=\code{m},
		description=a coding function or variable,
		type=notation
	}
	\glsadd{not:code}

% ==============================================================================
% Continuity Degree
% ==============================================================================
	\newcommand{\contdeg}[1]{\ensuremath{\gls{not:contdeg}^{#1}}}
	\newglossaryentry{not:contdeg}
	{%
		name=\ensuremath{C},
		description=Degree of differentiablity
		type=notation
	}
	\glsadd{not:contdeg}
% ==============================================================================
% Geometric Continuity Degree
% ==============================================================================
	\newcommand{\contdeggeom}[1]{\ensuremath{\gls{not:contdeggeom}^{#1}}}
	\newglossaryentry{not:contdeggeom}
	{%
		name=\ensuremath{G},
		description=Geometric degree of differentiablity
		type=notation
	}
	\glsadd{not:contdeggeom}
% ==============================================================================
% differential
% ==============================================================================
	%TODO this operator is not in the nomenclature
	\newcommand{\der}{\ensuremath{\mathrm{d}}}

% ==============================================================================
% Scalar
% ==============================================================================
	\newglossaryentry{not:scalar}
	{%
		name=\ensuremath{m},
		description=a scalar,
		type=notation
	}
	\glsadd{not:scalar}

% ==============================================================================
% Vector
% ==============================================================================
	\renewcommand{\vec}[1]{\ensuremath{\boldsymbol{#1}}}
	\newglossaryentry{not:vec}
	{%
		name=\vec{m}\ifdraft{: \textbackslash{} vec}{},
		description=a vector,
		type=notation
	}
	\glsadd{not:vec}

% ==============================================================================
% Projection
% ==============================================================================
	\newcommand{\project}[2]{\ensuremath{{}^{#2}\!{#1}}}
	\newglossaryentry{not:project}
	{%
		name=\project{\vec{m}}{n}\ifdraft{: \textbackslash{} project\{to\}}{},,
		description=projection of vector \vec{m} in frame $n$,
		type=notation
	}
	\glsadd{not:project}

% ==============================================================================
% Skew Matrix
% ==============================================================================
	\newcommand{\skewmat}[1]{\ensuremath{\widehat{#1}}}
	%\newcommand{\skewmat}[1]{\ensuremath{\left[{#1}\right]_{\mathrm{X}}}}
	\newglossaryentry{not:skewmat}
	{%
		name=\skewmat{\vec{m}}\ifdraft{: \textbackslash{} skew}{},
		description=the skew matrix associated with the vector \vec{m},
		type=notation
	}
	\glsadd{not:skewmat}

% ==============================================================================
% Matrix
% ==============================================================================
	\newcommand{\mat}[1]{\ensuremath{\boldsymbol{\mathrm{#1}}}}
	\newglossaryentry{not:mat}
	{%
		name=\mat{M}\ifdraft{: \textbackslash{} mat}{},
		description=a matrix,
		type=notation
	}
	\glsadd{not:mat}

% ==============================================================================
% Pseudo Inverse
% ==============================================================================
	\newcommand{\pseudoinv}[1]{\ensuremath{{#1}^{+}}}
	\newglossaryentry{not:pseudoinv}
	{%
		name=\pseudoinv{\mat{M}}\ifdraft{: \textbackslash{} pseudoinv}{},
		description=the pseudo inverse of matrix \mat{M},
		type=notation
	}
	\glsadd{not:pseudoinv}

% ==============================================================================
% Vector Space
% ==============================================================================
	\newcommand{\vecspace}[1]{\ensuremath{\mathscr{#1}}}
	\newglossaryentry{not:vecspace}
	{%
		name=\vecspace{M}\ifdraft{: \textbackslash{} vecspace}{},
		description=a vector space,
		type=notation
	}
	\glsadd{not:vecspace}

% ==============================================================================
% Vector Norm
% ==============================================================================
	\newcommand{\norm}[1]{\ensuremath{\left|\left|#1\right|\right|}}
	\newglossaryentry{not:norm}
	{%
		name=\norm{\vec{m}}\ifdraft{: \textbackslash{} norm}{},
		description=the norm of vector $\vec{m}$,
		type=notation
	}
	\glsadd{not:norm}

% ==============================================================================
% Set
% ==============================================================================
	\newcommand{\set}[1]{\ensuremath{\mathcal{#1}}}
	\newglossaryentry{not:set}
	{%
		name=\set{M}\ifdraft{: \textbackslash{} set}{},
		description=a set,
		type=notation
	}
	\glsadd{not:set}

% ==============================================================================
% Set Cardinality
% ==============================================================================
	\newcommand{\cardinality}[1]{\ensuremath{\left|#1\right|}}
	\newglossaryentry{not:cardinality}
	{%
		name=\cardinality{M}\ifdraft{: \textbackslash{} cardinality}{},
		description=the number of elements in set $M$,
		type=notation
	}
	\glsadd{not:cardinality}

% ==============================================================================
% Function
% ==============================================================================
	\newcommand{\func}[1]{\ensuremath{\mathrm{#1}}}
	\newglossaryentry{not:func}
	{%
		name=\func{m}\ifdraft{: \textbackslash{} func}{},
		description=a function,
		type=notation
	}
	\glsadd{not:func}

% ==============================================================================
% Vector Function
% ==============================================================================
	\newcommand{\vecfunc}[1]{\ensuremath{\boldsymbol{\mathrm{#1}}}}
	\newglossaryentry{not:vecfunc}
	{%
		name=\vecfunc{m}\ifdraft{: \textbackslash{} vecfunc}{},
		description=a vector function,
		type=notation
	}
	\glsadd{not:vecfunc}

% ==============================================================================
% Estimate
% ==============================================================================
	\newcommand{\estimate}[1]{\ensuremath{\widetilde{#1}}}
	\newglossaryentry{not:estimate}
	{%
		name=\estimate{m}\ifdraft{: \textbackslash{} estimate}{},
		description=an estimate of the quantity $m$,
		type=notation
	}
	\glsadd{not:estimate}

% ==============================================================================
% Approximation
% ==============================================================================
	\newcommand{\approximation}[1]{\ensuremath{\widetilde{#1}}}
	\newglossaryentry{not:approx}
	{%
		name=\approximation{m}\ifdraft{: \textbackslash{} approx}{},
		description=an approx of the quantity $m$,
		type=notation
	}
	\glsadd{not:approx}

% ==============================================================================
% nth Time Derivative
% ==============================================================================
	\newcommand{\tdern}[2]{\ensuremath{{#1}^{(#2)}}}
	\newglossaryentry{not:tdern}
	{%
		name=\tdern{m}{n}\ifdraft{: \textbackslash{} tdern}{},
		description=the nth time derivative of quantity $m$,
		type=notation
	}
	\glsadd{not:tdern}

% ==============================================================================
% Desired Value
% ==============================================================================
	\newcommand{\desired}[1]{\ensuremath{{#1}^{*}}}
	\newglossaryentry{not:desired}
	{%
		name=\desired{m}\ifdraft{: \textbackslash{} desired}{},
		description=the desired value of $m$,
		type=notation
	}
	\glsadd{not:desired}

% ==============================================================================
% Line
% ==============================================================================
	\newcommand{\vecline}[2]{\ensuremath{\overline{#1#2}}}
	\newglossaryentry{not:line}
	{%
		name=\ensuremath{\vecline{\vec{m_1}}{\vec{m_2}}}\ifdraft{: \textbackslash{} line}{},
		description=a line connecting points $m_1$ and $m_2$,
		type=notation
	}
	\glsadd{not:line}

% ==============================================================================
% Such That
% ==============================================================================
	\newglossaryentry{not:suchthat}
	{%
		name={\ensuremath{\backepsilon}},
		description=such that,
		type=notation
	}
	\newcommand{\suchthat}{\gls{not:suchthat}}

% ==============================================================================
% Floor and Ceiling
% ==============================================================================
	\DeclarePairedDelimiter{\ceil}{\lceil}{\rceil}
	\DeclarePairedDelimiter{\floor}{\lfloor}{\rfloor}
