\chapter{Conclusion}%
\label{chap:conclusion}

	The present thesis investigated methods for path planning and trajectory
	generation for \glspl{cdpr}. The final planning algorithm is based on a
	\gls{rrt} approach.

	Collision detection is done by modelling obstacles and the end-effector as
	unions of convex polyhedra. Cables are modelled as straight lines. Sagging
	is not taken into account. The approach also takes into account the static
	capacity margin along the trajectory as an optimisation criterion.

	Several post-processing steps were developed to generate simplified and
	smooth paths. The approach followed makes use of B-spline curves to ensure
	both the degree of differentiability of the trajectory, as well as the
	geometric degree of differentiability of the path.

	To this end, a software implementation of the algorithms discussed in this
	thesis was developed. The implementation is agnostic in the exact topology
	of the robot or in the amount and positions of obstacles in the workspace.
	Without modification of the source, the same code can be used to generate
	trajectories for a wide range of problems and a wide range of robots.

