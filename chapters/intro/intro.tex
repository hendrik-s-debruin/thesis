\chapter{Introduction}%
\label{chap:introduction}

	\glspl{cdpr} are a class of robot where the end-effector is suspended and
	supported by cables. The cables are connected to spools. Through actuation
	of the spools, the end-effector's pose in space may be controlled.

	Using cables for end-effector manipulation has some benefits over
	traditional serial and parallel robots. Cables are very lightweight. A
	\gls{cdpr} therefore tends to require smaller motors and are consequently
	inexpensive as compared to similar traditional robots. Furthermore, since
	cables are flexible, \glspl{cdpr} can be designed to be easily portable and
	reconfigurable. Due to their inexpensive and lightweight topology,
	\glspl{cdpr} can be made to cover very vast workspaces and are able to
	obtain very high velocities and accelerations.

	The added flexibility of this class of robot comes at a cost, however. When
	compared to traditional parallel robots, \glspl{cdpr} have the added
	constraint that all cables need to be in tension. That is, it is not
	possible to have a link push the end-effector as is the case with
	traditional robots. Furthermore, \glspl{cdpr} exhibit further new
	challenges, such as cable sagging and vibration.

	\glspl{cdpr} have seen few practical applications to date. However, due to
	their potential, they have started attracting more interest in the research
	community. A common problem in robotics is the generation of trajectories.
	When generating non-trivial trajectories for \glspl{cdpr}, care must be
	taken to avoid any obstacles in the workspace. This is complicated by the
	fact that the cables which support the robot can span large sections of the
	workspace.  Rotational trajectories can also cause cable-cable or
	cable-platform collisions. The proposed project is therefore to investigate
	ways of generating non-trivial trajectories and guaranteeing their safety.

	%%%%%%%%%%%%%%%%%%%%%%%%%%%%%%%%%%%%%%%%%%%%%%%%%%%%%%%%%%%%%%%%%%%%%%%%%%%%
	\todo{this paragraph must be removed for the thesis. A corresponding
	paragraph is given in the literature review, but is commented out for the
	literature review report. Uncomment later and delete this}

	The literature study for the present report has three main sections.
	Chapter~\ref{sec:trajectory_generation} reviews techniques for generating
	trajectories that meet certain criteria. Chapter~\ref{sec:path_planning}
	gives an overview of path planning methodologies. Finally,
	Chapter~\ref{sec:modelling_of_cable_driven_parallel_robots} gives an
	overview of the important aspects in modelling \glspl{cdpr}.

	A proposed work plan can be found in Chapter~\ref{sec:proposed_plan}.
	%%%%%%%%%%%%%%%%%%%%%%%%%%%%%%%%%%%%%%%%%%%%%%%%%%%%%%%%%%%%%%%%%%%%%%%%%%%%
