\section{End-Effector-Obstacle Collisions}%
\label{sec:end_effector_obstacle_collisions}

	Collisions between the end-effector and obstacles are detected using the
	well-known
	\gls{sat}~\cite{bib:planning:hierarchical_structure_for_rapid_interference_detection}
	theorem. \gls{sat} is not the fastest collision detection algorithm for two
	convex polyhedra. A superior algorithm may be found
	in~\cite{bib:planning:detecting_intersections_between_convex_polyhedra}.
	However, \gls{sat} is considerably simpler to implement. Furthermore, for
	the planning problems addressed by the current thesis, there are few
	obstacles that are modelled in such a way as to have few faces each. Under
	these conditions, \gls{sat} was found to perform sufficiently fast to not
	cause any bottle-neck in the overall architecture.

	For the sake of completeness, the algorithm is briefly described in
	Algorithm~\ref{alg:sat}. Essentially, it looks for a line that passes
	between the objects without intersecting them. It is sufficient to only test
	the normals of the faces of both objects.

	\begin{algorithm}[ht]
		\caption{Separating Axis Theorem Collision
		Detection~\cite{bib:planning:hierarchical_structure_for_rapid_interference_detection}}%
		\label{alg:sat}
		\begin{algorithmic}[1]
			\Procedure{sat}{$\robot$, $\obstacle$}
				\State{}Put the normalised normal $n$ of every face $\face \in \robot\cup\obstacle$ into a queue $\queue$
				\ForAll{$n \in \queue$}
					\ForAll{vertex $\vertex \in \robot$}
						\State{}Project $\vertex$ onto $n$, keeping track of the highest and lowest values
					\EndFor{}
					\ForAll{vertex $\vertex \in \obstacle$}
						\State{}Project $\vertex$ onto $n$, keeping track of the highest and lowest values
					\EndFor{}
					\If{the intervals do not overlap}
						\State{} Return $\code{no\_collision}$
					\EndIf{}
				\EndFor{}
				\State{}Return $\code{collision}$
			\EndProcedure{}
		\end{algorithmic}
	\end{algorithm}
