\section{Cable-Cable Collisions}%
\label{sec:cable_cable_collisions}

	To facilitate collision detection, the cables are modelled as parametric
	lines. As a brief recall, a line $\linevec$ between two points $\point_A$
	and $\point_B$ may be written in parametric form as:

	\begin{equation}
		\linevec = \point_A + \timenorm(\point_B - \point_A)
		\label{eq:parametric_line}
	\end{equation}

	Where $\timenorm \in [0, 1]$.

	To detect whether two cables are in collision, consider
	Figure~\ref{fig:cable_cable_collision_detection}.

	\begin{figure}[hb]
		\centering
		\def\svgwidth{\columnwidth}
		\import{res/img/}{cable_collision_detection.pdf_tex}
		\caption{Cable-Cable Collision Detection}
		\label{fig:cable_cable_collision_detection}
	\end{figure}

	The equations of the cables in the figure are:

	\begin{equation}
		\cable_i = \distalanchor_i + \timenorm_i(\proximalanchor_i -
		\distalanchor_i)
	\end{equation}

	for $i = \{1, 2\}$. Considering for now the two-dimensional case and
	assuming the cables are of infinite length, their intersection point may be
	found by solving:

	\begin{align}
		\cable_1 &= \cable_2 \\
		\distalanchor_1 + \timenorm_1(\proximalanchor_1 - \distalanchor_1) &=
			\distalanchor_2 + \timenorm_2(\proximalanchor_2 - \distalanchor_2)
	\end{align}

	This may be put into matrix form:

	\begin{align}
		\left[
			\begin{matrix}
				\proximalanchor_1 - \distalanchor_1 &
				\distalanchor_2 - \proximalanchor_2
			\end{matrix}
		\right]
		\left[
			\begin{matrix}
				\timenorm_1 \\
				\timenorm_2 \\
			\end{matrix}
		\right]
		&=
		\left[
			\begin{matrix}
				\distalanchor_2 - \distalanchor_1
			\end{matrix}
		\right]
		%
		\\
		%
		\mat{A}\vec{\timenorm} &= \vec{b}%
		\label{eq:cable_cable_interference}
	\end{align}
	\todo{Matrix symbols are not in nomenclature, but are not referenced
	in another section. Temporary symbols}

	In the two-dimensional case, we have that $\dim\mat{A} = 2\times2$ and
	$\dim\vec{\timenorm} = \dim\vec{b} = 2\times1$.
	Equation~\ref{eq:cable_cable_interference} can thus be solved by simply
	taking:

	\begin{equation}
		\vec{\timenorm} = {\mat{A}}^{-1}\vec{b}
	\end{equation}

	Assuming for now zero-width, the cables are in collision if and only if
	$\timenorm_i \in [0, 1]$.

	In the three-dimensional case, infinite lines could lie in different planes.
	This is reflected by the fact that $\dim\mat{A} = 3\times2$,
	$\dim\vec{\timenorm} = 2\times1$ and $\dim\vec{b} = 2\times1$. The system
	can therefore not be solved by taking the inverse of $\mat{A}$. Instead, the
	closest points on the two cables can be found by solving
	Equation~\ref{eq:cable_cable_interference} in the least squares sense:

	\begin{equation}
		\vec{\timenorm} = \pseudoinv{\mat{A}}\vec{b}
	\end{equation}

	Note that this will give the closest points on infinite lines. To find the
	closest points on the finite cables, it is therefore necessary to clamp the
	values in $\vec{\timenorm}$:

	\begin{equation}
		\timenorm_i =
		\begin{cases}
			0, \quad \timenorm_i < 0 \\
			1, \quad \timenorm_i > 1 \\
			\timenorm_i, \quad \text{otherwise}
		\end{cases}
	\end{equation}

	The cables are considered to be in collision if the nearest point is closer
	than some predefined tolerance:

	\begin{equation}
		\norm{\cable_1(\timenorm_1) - \cable_2(\timenorm_2)} < \tol
	\end{equation}

	A safety margin can be ensured by choosing the tolerance sufficiently larger
	than the diameter of a single cable.

	Note that some \gls{cdpr} designs place either the proximal or distal anchor
	of two or more cables at the same point to reduce the risk of
	collision~\cite{bib:cdpr:cable_driven_parallel_robots_theory_and_application}.
	For such cases, cables can be considered collision-free if:

	\begin{equation}
		\timenorm_i = 0 \quad\text{or}\quad \timenorm_i = 1
	\end{equation}
