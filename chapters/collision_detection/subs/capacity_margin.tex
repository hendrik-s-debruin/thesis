\section{Capacity Margin}%
\label{sec:capacity_margin}

	In general, the cables of a \gls{cdpr} have upper and lower bounds on their
	tensions. These bounds limit the set of wrenches at the end-effector that
	the cables can support. This set of available wrenches $\setofavwrenches$ is
	also pose-dependent. Furthermore, for a given application, it is possible to
	define a set of required wrenches $\setofreqwrenches$ that the end-effector
	must support. A pose $\pose$ can be considered valid if the set of available
	wrenches contains the set of required wrenches:

	\begin{equation}
		\setofavwrenches \supseteq \setofreqwrenches
		\label{eq:wrench_set_requirement}
	\end{equation}

	The capacity
	margin~\cite{bib:cdpr:measuring_how_well_a_structure_supports_varying_external_wrenches}
	$\capacitymargin$ is a scalar index that measures the
	degree to which Equation~\ref{eq:wrench_set_requirement} is true. It is
	defined such that:

	\begin{align}
		\begin{split}
			\capacitymargin \geq 0, \quad \setofavwrenches \supseteq \setofreqwrenches \\
			\capacitymargin < 0, \quad \text{otherwise}
		\end{split}
	\end{align}

	The capacity margin may be calculated using the hyperplane shifting
	method~\cite{bib:cdpr:on_the_ability_of_a_cable_driven_robot_to_generate_a_prescribed_set_of_wrenches}.

	% The algorithm for calculating the capacity margin was implemented in C++
	% using the hyperplane shifting method described in\todo{ref hyperplane
	% shifting paper}. The reader is referred to \todo{ref paper list} for the
	% algorithms.

	Although the capacity margin does not test for collisions, it is included in
	the collision-detection module. This is because a pose $\pose$ for which
	$\capacitymargin < 0$ cannot support the required wrenches for the current
	application. This implies that the pose should be considered invalid. From a
	planning perspective, a pose which has a negative capacity margin is
	therefore considered to be in collision.
