\section{Cable-End-Effector Collisions}%
\label{sec:cable_end_effector_collisions}

	Similarly to obstacles, the end-effector $\robot$ is modelled as a convex
	polyhedron. As a result, the same procedure of
	Section~\ref{sec:cable_obstacle_collisions} can be largely applied for this
	case.

	One small difference is that the cables are always touching the
	end-effector. Therefore, simply following the procedure from before would
	result in collisions always being detected. This is avoided by adjusting
	Equation~\ref{eq:cable_obstacle_collision_condition} slightly to obtain:

	\begin{equation}
		\begin{cases}
			\timenorm_1 &\in [0, 1)\\
			\cable(\timenorm_1) &\in \face_i
		\end{cases}
		\quad \forall i \in \{1, \ldots, \norm{\robot}\}
		\label{eq:cable_robot_collision_condition}
	\end{equation}
