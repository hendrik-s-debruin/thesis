\section{Cable-Obstacle Collisions}%
\label{sec:cable_obstacle_collisions}

	The approach used for cable-obstacle collision detection is based on the
	intersection of parametric lines and parametric planes. Similarly to the
	parametric line in Equation~\ref{eq:parametric_line}, a parametric plane
	$\plane$ through three points $\point_1, \point_2, \point_3$ can be
	expressed as:

	\begin{equation}
		\plane = \point_1 + \timenorm_1(\point_2 - \point_1) +
		\timenorm_2(\point_3 - \point_1)
	\end{equation}

	Where $\tau_1, \tau_2 \in \Re$.

	To detect whether a cable is in collision with an obstacle, consider
	Figure~\ref{fig:cable_obstacle_collision_detection}.

	\begin{figure}[hbt]
		%\missingfigure{Cable Obstacle Collision}
		\centering
		\def\svgwidth{\columnwidth}
		\import{res/img/}{cable_obstacle_collision.pdf_tex}
		\caption{Cable-Obstacle Collision Detection}%
		\label{fig:cable_obstacle_collision_detection}
	\end{figure}

	A convex polyhedron object $\obstacle$ consists of multiple faces,
	${\face}_i$. A cable is in collision with such an object if it intersects
	with any one of these faces.  The following procedure is therefore run for
	every face of the polyhedron.

	To detect a collision between a face and a cable, the face is first
	approximated as an infinite parametric plane through any three vertices
	$\vertex_i$ on the face. The equations for the cable and infinite plane are:

	\todo{I changed the direction of the cable here to be consistent with the
	standard. Check that the equations are still correct}
	\begin{align}
		\cable &= \distalanchor + \timenorm_1(\proximalanchor - \distalanchor)\\
		\plane &= \vertex_1 + \timenorm_2(\vertex_2 - \vertex_1) +
		\timenorm_3(\vertex_3 - \vertex_1)
	\end{align}

	Equating the points and manipulating yields the following expressions:

	\begin{align}
		\cable &= \plane
		%%
		\\
		%%
		\left[
			\begin{matrix}
				\proximalanchor - \distalanchor &
				\vertex_1 - \vertex_2 &
				\vertex_1 - \vertex_2
			\end{matrix}
		\right]
		\left[
			\begin{matrix}
				\timenorm_1 \\
				\timenorm_2 \\
				\timenorm_3
			\end{matrix}
		\right]
		&=
		\left[
			\begin{matrix}
				\vertex_1 - \distalanchor
			\end{matrix}
		\right]
		%%
		\\
		%%
		\mat{A}\vec{\timenorm} &= \vec{b}
	\end{align}
	\todo{temp symbols not referenced later or in nomenclature}

	Since $\dim\mat{A} = 3\times3$, $\dim\vec{\timenorm} = 3\times1$ and
	$\dim\vec{b} = 3\times1$, the intersection of an infinite cable with an
	infinite plane is simply:

	\begin{equation}
		\vec{\timenorm} = {\mat{A}}^{-1}\vec{b}
	\end{equation}

	Now, a necessary condition for collision is that:

	\begin{equation}
		\timenorm_1 \in [0, 1]
	\end{equation}

	If $\timenorm_1 \not\in [0,1]$, it means that the cable stops before hitting
	the plane. In such a case, the collision detection algorithm may return
	early.

	For the special case where the face of the polyhedron is a rectangle, a
	sufficient condition for collision is $\timenorm_i \in [0, 1], \quad i =
	\{1, 2, 3\}$.  However, in the general case, when $\timenorm_1 \in [0, 1]$,
	it is not sufficient to test that $\timenorm_2,\timenorm_3 \in [0, 1]$. This
	is because the section of the infinite plane $\plane$ described by the range
	$\timenorm_2, \timenorm_3 \in [0, 1]$ is the regular parallelogram defined
	by the vertices $\point_1, \point_2, \point_3$.
	Figure~\ref{fig:point_in_plane} shows a simple example why this is the case.
	As can be seen in the figure, point $\point_4$ is inside the parametric
	plane $\plane$ described by the vertices $\point_1$, $\point_2$ and
	$\point_3$, even though $\timenorm_3 \not\in [0, 1]$.

	\begin{figure}[hbt]
		%\missingfigure{Cable Obstacle Collision}
		\centering
		\def\svgwidth{\columnwidth}
		\import{res/img/}{point_in_plane.pdf_tex}
		\caption{Point in $\plane$}%
		\label{fig:point_in_plane}
	\end{figure}

	The cable intersects the plane containing the current face of the polyhedron
	at point $\cable(\timenorm_1)$. It is therefore sufficient to check that
	$\cable(\timenorm_1) \in \face_i$. This is done by use of the logical
	predicate in Equation~\ref{eq:collision_point_with_obstacle},
	page~\pageref{eq:collision_point_with_obstacle}.

	The complete condition for collision is then given by:

	\begin{equation}
		\begin{cases}
			\timenorm_1 &\in [0, 1]\\
			\cable(\timenorm_1) &\in \face_i
		\end{cases}
		\quad \forall i \in \{1, \ldots, \norm{\obstacle}\}
		\label{eq:cable_obstacle_collision_condition}
	\end{equation}

	Where the norm denotes the number of faces of the obstacle $\obstacle$.
