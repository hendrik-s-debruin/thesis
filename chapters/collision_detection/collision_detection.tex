\chapter{Collision Detection}%
\label{chap:collision_detection}

	% Part~\ref{part:path_planning_approach} discusses in more detail the path
	% planning approach followed in the current thesis. This chapter discusses the
	% collision detection algorithms used, while later chapters discuss the
	% planning and post processing steps.

	There are several types of collision that may occur in a \glspl{cdpr}. These
	are:

	\begin{enumerate}

		\item Cable-Cable collisions

		\item Cable-Obstacle collisions

		\item Cable-End-Effector collisions

		\item End-Effector-Obstacle collisions

	\end{enumerate}

	The following assumptions were made for the collision-detection
	algorithms:

	\begin{enumerate}

		\item

			Cables are modelled as straight lines. If the tension in the cables
			is sufficiently high, this assumption is satisfied. The current
			collision-detection algorithms make no attempt to detect collisions
			in sagging cables.

		\item

			The end-effector and the obstacles are approximated as convex
			polyhedra. To ensure a safety margin, obstacles are usually modelled
			to be slightly larger than they are in reality. This justifies the
			use of convex  shapes to approximate them. For cases where modelling
			an obstacle as a convex shape is unnecessarily wasteful, a
			non-convex obstacle can be built up out of the union of convex
			obstacles.

		\item

			The capacity
			margin (briefly defined in Section~\ref{sec:capacity_margin})
			is considered in the static case only. Dynamic effects on the
			capacity margin are neglected.

	\end{enumerate}

	The current chapter discusses the algorithms used to detect these classes of
	collisions.

	\section{Cable-Cable Collisions}%
\label{sec:cable_cable_collisions}

	To facilitate collision detection, the cables are modelled as parametric
	lines. As a brief recall, a line $\linevec$ between two points $\point_A$
	and $\point_B$ may be written in parametric form as:

	\begin{equation}
		\linevec = \point_A + \timenorm(\point_B - \point_A)
		\label{eq:parametric_line}
	\end{equation}

	Where $\timenorm \in [0, 1]$.

	To detect whether two cables are in collision, consider
	Figure~\ref{fig:cable_cable_collision_detection}.

	\begin{figure}[hb]
		\centering
		\def\svgwidth{\columnwidth}
		\import{res/img/}{cable_collision_detection.pdf_tex}
		\caption{Cable-Cable Collision Detection}
		\label{fig:cable_cable_collision_detection}
	\end{figure}

	The equations of the cables in the figure are:

	\begin{equation}
		\cable_i = \distalanchor_i + \timenorm_i(\proximalanchor_i -
		\distalanchor_i)
	\end{equation}

	for $i = \{1, 2\}$. Considering for now the two-dimensional case and
	assuming the cables are of infinite length, their intersection point may be
	found by solving:

	\begin{align}
		\cable_1 &= \cable_2 \\
		\distalanchor_1 + \timenorm_1(\proximalanchor_1 - \distalanchor_1) &=
			\distalanchor_2 + \timenorm_2(\proximalanchor_2 - \distalanchor_2)
	\end{align}

	This may be put into matrix form:

	\begin{align}
		\left[
			\begin{matrix}
				\proximalanchor_1 - \distalanchor_1 &
				\distalanchor_2 - \proximalanchor_2
			\end{matrix}
		\right]
		\left[
			\begin{matrix}
				\timenorm_1 \\
				\timenorm_2 \\
			\end{matrix}
		\right]
		&=
		\left[
			\begin{matrix}
				\distalanchor_2 - \distalanchor_1
			\end{matrix}
		\right]
		%
		\\
		%
		\mat{A}\vec{\timenorm} &= \vec{b}%
		\label{eq:cable_cable_interference}
	\end{align}

	In the two-dimensional case, we have that $\dim\mat{A} = 2\times2$ and
	$\dim\vec{\timenorm} = \dim\vec{b} = 2\times1$.
	Equation~\ref{eq:cable_cable_interference} can thus be solved by simply
	taking:

	\begin{equation}
		\vec{\timenorm} = {\mat{A}}^{-1}\vec{b}
	\end{equation}

	Assuming for now zero-width, the cables are in collision if and only if
	$\timenorm_i \in [0, 1]$.

	In the three-dimensional case, infinite lines could lie in different planes.
	This is reflected by the fact that $\dim\mat{A} = 3\times2$,
	$\dim\vec{\timenorm} = 2\times1$ and $\dim\vec{b} = 2\times1$. The system
	can therefore not be solved by taking the inverse of $\mat{A}$. Instead, the
	closest points on the two cables can be found by solving
	Equation~\ref{eq:cable_cable_interference} in the least squares sense:

	\begin{equation}
		\vec{\timenorm} = \pseudoinv{\mat{A}}\vec{b}
	\end{equation}

	Note that this will give the closest points on infinite lines. To find the
	closest points on the finite cables, it is therefore necessary to clamp the
	values in $\vec{\timenorm}$:

	\begin{equation}
		\timenorm_i =
		\begin{cases}
			0, \quad \timenorm_i < 0 \\
			1, \quad \timenorm_i > 1 \\
			\timenorm_i, \quad \text{otherwise}
		\end{cases}
	\end{equation}

	The cables are considered to be in collision if the nearest point is closer
	than some predefined tolerance:

	\begin{equation}
		\norm{\cable_1(\timenorm_1) - \cable_2(\timenorm_2)} < \tol
	\end{equation}

	A safety margin can be ensured by choosing the tolerance sufficiently larger
	than the diameter of a single cable.

	Note that some \gls{cdpr} designs place either the proximal or distal anchor
	of two or more cables at the same point to reduce the risk of
	collision~\cite{bib:cdpr:cable_driven_parallel_robots_theory_and_application}.
	For such cases, cables can be considered collision-free if:

	\begin{equation}
		\timenorm_i = 0 \quad\text{or}\quad \timenorm_i = 1
	\end{equation}

	\section{Cable-Obstacle Collisions}%
\label{sec:cable_obstacle_collisions}

	The approach used for cable-obstacle collision detection is based on the
	intersection of parametric lines and parametric planes. Similarly to the
	parametric line in Equation~\ref{eq:parametric_line}, a parametric plane
	$\plane$ through three points $\point_1, \point_2, \point_3$ can be
	expressed as:

	\begin{equation}
		\plane = \point_1 + \timenorm_1(\point_2 - \point_1) +
		\timenorm_2(\point_3 - \point_1)
	\end{equation}

	Where $\tau_1, \tau_2 \in \Re$.

	To detect whether a cable is in collision with an obstacle, consider
	Figure~\ref{fig:cable_obstacle_collision_detection}.

	\begin{figure}[hbt]
		%\missingfigure{Cable Obstacle Collision}
		\centering
		\def\svgwidth{\columnwidth}
		\import{res/img/}{cable_obstacle_collision.pdf_tex}
		\caption{Cable-Obstacle Collision Detection}%
		\label{fig:cable_obstacle_collision_detection}
	\end{figure}

	A convex polyhedron object $\obstacle$ consists of multiple faces,
	${\face}_i$. A cable is in collision with such an object if it intersects
	with any one of these faces.  The following procedure is therefore run for
	every face of the polyhedron.

	To detect a collision between a face and a cable, the face is first
	approximated as an infinite parametric plane through any three vertices
	$\vertex_i$ on the face. The equations for the cable and infinite plane are:

	\begin{align}
		\cable &= \distalanchor + \timenorm_1(\proximalanchor - \distalanchor)\\
		\plane &= \vertex_1 + \timenorm_2(\vertex_2 - \vertex_1) +
		\timenorm_3(\vertex_3 - \vertex_1)
	\end{align}

	Equating the points and manipulating yields the following expressions:

	\begin{align}
		\cable &= \plane
		%%
		\\
		%%
		\left[
			\begin{matrix}
				\proximalanchor - \distalanchor &
				\vertex_1 - \vertex_2 &
				\vertex_1 - \vertex_3
			\end{matrix}
		\right]
		\left[
			\begin{matrix}
				\timenorm_1 \\
				\timenorm_2 \\
				\timenorm_3
			\end{matrix}
		\right]
		&=
		\left[
			\begin{matrix}
				\vertex_1 - \distalanchor
			\end{matrix}
		\right]
		%%
		\\
		%%
		\mat{A}\vec{\timenorm} &= \vec{b}
	\end{align}

	Since $\dim\mat{A} = 3\times3$, $\dim\vec{\timenorm} = 3\times1$ and
	$\dim\vec{b} = 3\times1$, the intersection of an infinite cable with an
	infinite plane is simply:

	\begin{equation}
		\vec{\timenorm} = {\mat{A}}^{-1}\vec{b}
	\end{equation}

	Now, a necessary condition for collision is that:

	\begin{equation}
		\timenorm_1 \in [0, 1]
	\end{equation}

	If $\timenorm_1 \not\in [0,1]$, it means that the finite cable stops before
	hitting the plane. In such a case, the collision detection algorithm may
	return early.

	For the special case where the face of the polyhedron is a rectangle, a
	sufficient condition for collision is $\timenorm_i \in [0, 1], \quad i =
	\{1, 2, 3\}$.  However, in the general case, when $\timenorm_1 \in [0, 1]$,
	it is not sufficient to test that $\timenorm_2,\timenorm_3 \in [0, 1]$. This
	is because the section of the infinite plane $\plane$ described by the range
	$\timenorm_2, \timenorm_3 \in [0, 1]$ is the regular parallelogram defined
	by the vertices $\vertex_1, \vertex_2, \vertex_3$.
	Figure~\ref{fig:point_in_plane} shows a simple example why this is the case.
	As can be seen in the figure, point $\point$ is inside the parametric
	plane $\plane$ described by the vertices $\vertex_1$, $\vertex_2$ and
	$\vertex_3$, even though $\timenorm_3 \not\in [0, 1]$.

	\begin{figure}[hbt]
		%\missingfigure{Cable Obstacle Collision}
		\centering
		\def\svgwidth{\columnwidth}
		\import{res/img/}{point_in_plane.pdf_tex}
		\caption{Point in $\plane$}%
		\label{fig:point_in_plane}
	\end{figure}

	The cable intersects the plane containing the current face of the polyhedron
	at point $\cable(\timenorm_1)$. It is therefore sufficient to check that
	$\cable(\timenorm_1) \in \face_i$. This is done by use of the logical
	predicate in Equation~\ref{eq:collision_point_with_obstacle},
	page~\pageref{eq:collision_point_with_obstacle}.

	The complete condition for collision is then given by:

	\begin{equation}
		\begin{cases}
			\timenorm_1 &\in [0, 1]\\
			\cable(\timenorm_1) &\in \face_i
		\end{cases}
		\quad \forall i \in \{1, \ldots, \norm{\obstacle}\}
		\label{eq:cable_obstacle_collision_condition}
	\end{equation}

	Where the norm denotes the number of faces of the obstacle $\obstacle$.

	\section{Cable-End-Effector Collisions}%
\label{sec:cable_end_effector_collisions}

	Similarly to obstacles, the end-effector $\robot$ is modelled as a convex
	polyhedron. As a result, the same procedure of
	Section~\ref{sec:cable_obstacle_collisions} can be largely applied for this
	case.

	One small difference is that the cables are always touching the
	end-effector. Therefore, simply following the procedure from before would
	result in collisions always being detected. This is avoided by adjusting
	Equation~\ref{eq:cable_obstacle_collision_condition} slightly to obtain:

	\begin{equation}
		\begin{cases}
			\timenorm_1 &\in [0, 1)\\
			\cable(\timenorm_1) &\in \face_i
		\end{cases}
		\quad \forall i \in \{1, \ldots, \norm{\robot}\}
		\label{eq:cable_robot_collision_condition}
	\end{equation}

	\section{End-Effector-Obstacle Collisions}%
\label{sec:end_effector_obstacle_collisions}

	Collisions between the end-effector and obstacles are detected using the
	well-known
	\gls{sat}~\cite{bib:planning:hierarchical_structure_for_rapid_interference_detection}
	theorem. \gls{sat} is not the fastest collision detection algorithm for two
	convex polyhedra. A superior algorithm may be found
	in~\cite{bib:planning:detecting_intersections_between_convex_polyhedra}.
	However, \gls{sat} is considerably simpler to implement. Furthermore, for
	the planning problems addressed by the current thesis, there are few
	obstacles that are modelled in such a way as to have few faces each. Under
	these conditions, \gls{sat} was found to perform sufficiently fast to not
	cause any bottle-neck in the overall architecture.

	For the sake of completeness, the algorithm is briefly described in
	Algorithm~\ref{alg:sat}. Essentially, it looks for a line that passes
	between the objects without intersecting them. It is sufficient to only test
	the normals of the faces of both objects.

	\begin{algorithm}[ht]
		\caption{Separating Axis Theorem Collision
		Detection~\cite{bib:planning:hierarchical_structure_for_rapid_interference_detection}}%
		\label{alg:sat}
		\begin{algorithmic}[1]
			\Procedure{sat}{$\robot$, $\obstacle$}
				\State{}Put the normalised normal $n$ of every face $\face \in \robot\cup\obstacle$ into a queue $\queue$
				\ForAll{$n \in \queue$}
					\ForAll{vertex $\vertex \in \robot$}
						\State{}Project $\vertex$ onto $n$, keeping track of the highest and lowest values
					\EndFor{}
					\ForAll{vertex $\vertex \in \obstacle$}
						\State{}Project $\vertex$ onto $n$, keeping track of the highest and lowest values
					\EndFor{}
					\If{the intervals do not overlap}
						\State{} Return $\code{no\_collision}$
					\EndIf{}
				\EndFor{}
				\State{}Return $\code{collision}$
			\EndProcedure{}
		\end{algorithmic}
	\end{algorithm}

	\section{Capacity Margin}%
\label{sec:capacity_margin}

	In general, the cables of a \gls{cdpr} have upper and lower bounds on their
	tensions. These bounds limit the set of wrenches at the end-effector that
	the cables can support. This set of available wrenches $\setofavwrenches$ is
	also pose-dependent. Furthermore, for a given application, it is possible to
	define a set of required wrenches $\setofreqwrenches$ that the end-effector
	must support. A pose $\pose$ can be considered valid if the set of available
	wrenches contains the set of required wrenches:

	\begin{equation}
		\setofavwrenches \supseteq \setofreqwrenches
		\label{eq:wrench_set_requirement}
	\end{equation}

	The capacity
	margin~\cite{bib:cdpr:measuring_how_well_a_structure_supports_varying_external_wrenches}
	$\capacitymargin$ is a scalar index that measures the
	degree to which Equation~\ref{eq:wrench_set_requirement} is true. It is
	defined such that:

	\begin{align}
		\begin{split}
			\capacitymargin \geq 0, \quad \setofavwrenches \supseteq \setofreqwrenches \\
			\capacitymargin < 0, \quad \text{otherwise}
		\end{split}
	\end{align}

	The capacity margin may be calculated using the hyperplane shifting
	method~\cite{bib:cdpr:on_the_ability_of_a_cable_driven_robot_to_generate_a_prescribed_set_of_wrenches}.

	% The algorithm for calculating the capacity margin was implemented in C++
	% using the hyperplane shifting method described in\todo{ref hyperplane
	% shifting paper}. The reader is referred to \todo{ref paper list} for the
	% algorithms.

	Although the capacity margin does not test for collisions, it is included in
	the collision-detection module. This is because a pose $\pose$ for which
	$\capacitymargin < 0$ cannot support the required wrenches for the current
	application. This implies that the pose should be considered invalid. From a
	planning perspective, a pose which has a negative capacity margin is
	therefore considered to be in collision.

	\section{Overall Collision Detection Algorithm}%
\label{sec:overall_collision_detection_algorithm}

	The collision detection algorithms described in this chapter execute
	quickly. However, since they can potentially be called a few thousand times
	during the initial path search, it is beneficial to organise them in such a
	way that the algorithms that are more likely to detect a collision are
	executed before those that are less likely. This can lead to potential
	speed-ups during the execution  of the code.  The overall meta-algorithm for
	collision detection is reported in
	Algorithm~\ref{alg:overall_collision_detection}.

	\begin{algorithm}[ht]
		\caption{Overall Collision Detection}%
		\label{alg:overall_collision_detection}
		\begin{algorithmic}[1]
			\Procedure{in\_collision}{}
				\ForAll{\cable}
					\ForAll{\obstacle}
						\If{\code{cable\_obstacle\_collision(\cable, \obstacle)}}
							\State{}\code{return true}
						\EndIf{}
					\EndFor{}
				\EndFor{}
				\ForAll{\obstacle}
					\If{\code{endeffector\_obstacle\_collision(\robot, \obstacle)}}
						\State{}\code{return true}
					\EndIf{}
				\EndFor{}
				\ForAll{$\cable_{\indexi}$}
					\ForAll{$\cable_{\indexj} \neq \cable_{\indexi}$}
						\If{$\code{cable\_cable\_collision}(\cable_{\indexi}, \cable_{\indexj})$}
					 		\State{}\code{return true}
						\EndIf{}
					\EndFor{}
				\EndFor{}
				\ForAll{\cable}
					\If{\code{cable\_endeffector\_collision(\cable, \robot)}}
						\State{}\code{return true}
					\EndIf{}
				\EndFor{}
				\State{}$\capacitymargin = \robot(\pose)\code{.capacity\_margin()}$
				\If{$\capacitymargin < 0$}
					\State{}\code{return true}
				\EndIf{}
				\State{}\code{return false}
			\EndProcedure{}
		\end{algorithmic}
	\end{algorithm}

