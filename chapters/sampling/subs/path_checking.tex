\section{Path Checking}%
\label{sec:path_checking}

	\begin{sloppypar}

		There is no guarantee that a sampled pose $\pose_\text{sampled}$ can
		be connected directly to its nearest neighbour
		$\pose_\text{neighbour} \in \topologicalgraph$.
		Figure~\ref{fig:nearest_neighbour} shows one such case. In the figure,
		the green dot represents the last sampled pose. There exists no
		straight-line collision-free path from its nearest neighbour, indicated
		by the red line.

		\begin{figure}[hb]
			\centering
			\def\svgwidth{0.75\columnwidth}
			\import{res/img/}{nearest_neighbour.pdf_tex}
			\caption{Invalid nearest neighbour}%
			\label{fig:nearest_neighbour}
		\end{figure}

		For this reason, the calls to
		$\code{farthest\_collision\_free\_point}$ in lines
		Lines~\ref{alg:sampling_planning_overview:farthest_collision_free_point_sampled}
		and~\ref{alg:sampling_planning_overview:farthest_collision_free_point_goal}
		of Algorithm~\ref{alg:sampling_planning_overview} attempt to
		guarantee that there is no collision between these points.

	\end{sloppypar}

	This is done by discretising points between $\pose_\text{neighbour}$ and
	$\pose_\text{sampled}$ in a straight-line fashion and returning the
	point farthest from $\pose_\text{neighbour}$ before any collision is
	detected. This is summarised in
	Algorithm~\ref{alg:farthest_collision_free_point}.

	\begin{algorithm}[ht]
		\caption{Farthest Collision Free Point}%
		\label{alg:farthest_collision_free_point}
		\begin{algorithmic}[1]
			\Procedure{Farthest\_Collision\_Free\_Point}{$\pose_1, \pose_2$}
				\State{}$\code{dist\_to\_travel} = \dist(\pose_1, \pose_2)$
				\State{}$\code{dist\_travelled} = 0$
				\State{}$\pose_\text{intermediate} = \pose_1$
				\State{}$\pose_\text{prev-intermediate} = \pose_1$
				\While{$\code{dist\_travelled} < \code{dist\_to\_travel}$}
					\State{}
						\(
							\pose_\text{intermediate} =
							\pose_1 +
							\code{dist\_travelled}/\code{dist\_to\_travel}
							(
								\pose_2 - \pose_1
							)
						\)
					\If{\robot($\pose_\text{intermediate}) \in \configurationspace_{\obstacleregion}$}
						\State{}$\pose_\text{intermediate} =
							\pose_\text{prev-intermediate}$
						\State{}break
					\EndIf{}
					\State{}$\code{dist\_travelled} =
						\code{dist\_travelled} + \Delta\configuration$
					\State{}$\pose_\text{prev-intermediate} =
						\pose_\text{intermediate}$
				\EndWhile{}
				\If{$\code{dist\_travelled} > \code{dist\_to\_travel}$}
					\State{}return $\pose_2$
				\EndIf{}
				\State{}return $\pose_\text{intermediate}$
			\EndProcedure{}
		\end{algorithmic}
	\end{algorithm}

	The term $\Delta\configuration$ in
	Algorithm~\ref{alg:farthest_collision_free_point} is an increment that
	determines the resolution at which paths are checked. Its value must be
	chosen such that the algorithm executes quickly, but does not jump over
	obstacles. Its value essentially determines a sphere of radius
	$\Delta\configuration$ around every sampled pose $\pose$ within which it
	is not certain if there is a collision or not.

	% This uncertainty is negated by creating virtual obstacles that are
	% larger than the actual obstacles by a value
	% greater than $\Delta\configuration$. That way, if there is a collision
	% in the uncertain points between $\pose_1$ and $\pose_2$, it will be a
	% collision with a virtual obstacle, but not with the real obstacle.
	% \todo{When this is implemented, discuss it better}
