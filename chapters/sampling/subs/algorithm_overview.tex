\section{Sampling Algorithm Overview}%
\label{sec:algorithm_overview}

	% Due to the topology of the class of \gls{cdpr} considered, the sampling
	% algorithm does not need to sample in configuration space, but may instead
	% sample poses directly. This is due to the fact that there is a one-to-one
	% correlation between end-effector pose and robot configuration.

	The first step of the algorithm is to build a topological graph
	$\topologicalgraph$ whose vertices contain collision-free end-effector
	poses. The edges of the graph represent paths between these poses that are
	also free of collision.  After $\topologicalgraph$ is built, a path
	$\pathsym$ is found that connects the start pose $\pose_{\initial}$ to the
	final pose $\pose_{\final}$ by taking into account a custom-defined distance
	function $\dist(\pose_1, \pose_2)$ between two neighbour poses in
	$\topologicalgraph$. A schematic representation is shown in
	Figure~\ref{fig:path_search}. The left side of the figure shows a
	representation of $\topologicalgraph$ in configuration space, whereas the
	right side shows an as-of-yet inefficient path from the start to the goal.

	\begin{figure}[hb]
		\centering
		\def\svgwidth{\columnwidth}
		\import{res/img/}{path_search.pdf_tex}
		\caption{Initial Path Search}%
		\label{fig:path_search}
	\end{figure}


	The pose $\pose$ of the robot $\robot$ is described as a translation
	vector $\transvec$ combined with a quaternion $\quaternion$:

	\begin{equation}
		\pose = (\transvec, \quaternion)
	\end{equation}

	Furthermore, the algorithm defines a graph $\topologicalgraph$ of poses.
	During each iteration, the algorithm samples a new pose,
	$\pose_{\indexi}$ such that:

	\begin{equation}
		\robot(\pose_{\indexi}) \not\in
		\configurationspace_{\obstacleregion}
	\end{equation}

	and attempts to add it to the graph:

	\begin{equation}
		\topologicalgraph_{\indexi + 1} \leftarrow
			\topologicalgraph_{\indexi} \cup \pose_{\indexi}
	\end{equation}

	in such a way that:

	\begin{equation}
		\exists \pose_{\indexj} \in \topologicalgraph_{\indexi}
			\quad\suchthat\quad
			\vecline{\pose_{\indexi}}{\pose_{\indexj}} \notin
			\configurationspace_{\obstacleregion}
	\end{equation}

	This last line states that the new sampled pose may be connected to some
	other pose already in the graph in such a way that the path connecting
	the two poses is free of collisions.

	At the same time, at each iteration, the algorithm attempts to add the
	goal pose $\pose_{\goal}$ to $\topologicalgraph$. If this is
	successfully executed, then a path has been found and the algorithm
	terminates. A high-level overview of the algorithm can be found in
	Algorithm~\ref{alg:sampling_planning_overview}.

	\begin{algorithm}[ht]
		\caption{Sampling Planning Overview}%
		\label{alg:sampling_planning_overview}
		\begin{algorithmic}[1]
			\Procedure{Sample\_Search}{$\robot, \obstacle_1, \ldots, \obstacle_n$}
				\State{}$\topologicalgraph = \emptyset \cup \pose_{\initial}$
				\While{$\pose_{\goal} \notin \topologicalgraph$}
					\Repeat{}
						\State{}$\pose_\text{sampled} \leftarrow
						\code{sample\_pose}$\label{alg:sampling_planning_overview:sample_pose}
					\Until{$\robot(\pose) \notin \configurationspace_{\obstacleregion}$}
					\State{}
						\(
							\pose_{\text{neighbour}} =
							\argmin_{\pose_{\indexi} \in
							\topologicalgraph}\dist(\pose_\text{sampled}, \pose_{\indexi})
						\)
					\State{}$\pose_\text{new} =
						\code{farthest\_collision\_free\_point}(\pose_{\text{sampled}},
						\pose_{\text{neighbour}})$\label{alg:sampling_planning_overview:farthest_collision_free_point_sampled}
					\State{}$\code{connect}(\pose_\text{neighbour},
						\pose_\text{new})$
					\State{}$\pose_\text{new} =
						\code{farthest\_collision\_free\_point}(\pose_{\text{new}},
						\pose_{\text{\goal}})$\label{alg:sampling_planning_overview:farthest_collision_free_point_goal}
					\If{$\pose_\text{new} == \pose_{\goal}$}
						\State{}$\code{connect}(\pose_{\goal},
							\pose_\text{new})$
					\EndIf{}
				\EndWhile{}
			\EndProcedure{}
		\end{algorithmic}
	\end{algorithm}
