\section{Distance Measure}%
\label{sec:distance_measure}

	Both Algorithm~\ref{alg:sampling_planning_overview}
	and~\ref{alg:farthest_collision_free_point} require to measure the
	distance between two poses, $\dist(\pose_1, \pose_2)$. This poses a
	problem, however, as a pose is a member of the special euclidean group,
	$\pose \in \specialEuclideanGroup{3}$. There is no one way to measure
	distances in this set and an arbitrary measure is often required. There
	is also the problem that the units used for rotations and translations
	differ.

	Several different measures are often used in these cases. One is to
	simply define the distance as the translational distance between two
	poses:

	\begin{equation}
		\dist(\pose_1, \pose_2) =
			\norm{%
				\pose_2.\transvec - \pose_1.\transvec
			}
		\label{eq:euclidean_distance}
	\end{equation}

	\todo{Add radius of gyration discussion here}

	This thesis proposes to instead measure distances in cable-space. This
	has the following advantages:

	\begin{itemize}

		\item

			Dimensionally homogeneous. Since each pose is related directly
			to the length of the cable-space vector, both the translation
			and orientation of a pose can be measured using meters.

		\item

			Direct correlation to actuators. The change in length of the
			cables relates directly to how much the actuators have to act on
			the cables. Using the cable-space vector as a distance measure
			means that the distance exactly encodes the amount of actuation
			required to move from one pose to another.
			\todo{wording}

	\end{itemize}

	The cable-space vector, $\cablelengths$, can be found from the indirect
	geometric model of the \gls{cdpr}:

	\begin{equation}
		\cablelengths = \invgeometricmodel(\pose)
	\end{equation}

	The distance function is then taken as:\todo{make sure this is correct,
	and implement it in the code}

	\begin{equation}
		\dist(\pose_1, \pose_2) =
			\norm{%
				\invgeometricmodel(\pose_2) - \invgeometricmodel(\pose_1)
			}
		\label{eq:distance_measure}
	\end{equation}

	\subsection{Incorporating the Capacity Margin}%
	\label{sec:incorporating_the_capacity_margin}

		It may be useful to include the value of an index, such as the capacity
		margin, in the distance calculation. For such cases
		Equation~\ref{eq:distance_measure} may be modified to obtain:

		\begin{equation}
			\dist_{\pathsym}(\pose_1, \pose_2) =
				-\gain_{\capacitymargin}
				\frac%
				{%
					\capacitymargin(\pose_2) - \capacitymargin(\pose_1)
				}
				{%
					%\int_{\pathsym} \der\pose
					%\int_0^1 \capacitymargin(\pathsym(\timenorm))\der\timenorm
					\forcemag_{\max}
				}
				+
				(1 - \gain_{\capacitymargin})
				\frac%
				{%
					\norm{%
						\invgeometricmodel(\pose_2) - \invgeometricmodel(\pose_1)
					}
				}
				{%
					\diag(\workspace)
				}
				\label{eq:distance_measure_capcity_margin}
		\end{equation}
		\todo{This has not yet been implemented in code}
		\todo{dimensionally homogeneous?}

		Here
		\(
			\dist_{\pathsym}
		\)
		measures the distance along the path $\pathsym$.
		\(
			\gain_{\capacitymargin} \in [0, 1]
		\)
		is a weighting factor that weighs the relative importance of the change
		in capacity margin for a longer path to the straight-line distance
		between two poses.
		\(
			\forcemag_{\max}
		\)
		is the maximum cable tension, and
		\(
			\diag(\workspace)
		\)
		is the diagonal corner-to-corner length of the workspace.

		The numerator of the first term measures the change in capacity margin
		between two poses and is measured in Newton. As described above, the
		numerator of the second term is measured in meters. As such, the
		denominators are introduced to normalise these terms and ensure that
		they are dimensionally homogeneous. This introduces a degree of
		arbitrariness in the distance function, which now becomes a
		dimensionless indicator function.

		The negative sign in front of the first term in
		Equation~\ref{eq:distance_measure_capcity_margin} is due to the fact
		that a path for which the capacity margin is higher should be considered
		more advantageous and therefore shorter.
