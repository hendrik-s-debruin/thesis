\section{Sample Strategy}%
\label{sec:sample_strategy}

	Line~\ref{alg:sampling_planning_overview:sample_pose} of
	Algorithm~\ref{alg:sampling_planning_overview} samples a pose from the
	configuration space of the robot in the following manner:

	\begin{enumerate}

		\item

			A translation vector $\transvec$ is drawn from a normal
			distribution such that:
			\label{item:sample_strategy:translation}

			\begin{enumerate}

				\item

					The median of the distribution is the centre of the
					workspace.

				\item

					The edges of the workspace are three standard deviations
					away from the median.
			\end{enumerate}

		\item

			An angle $\theta$ is drawn from a normal distribution with
			median $\pi/2$ and standard deviation $\pi/6$.

			\label{item:sample_strategy:angle}

		\item

			A quaternion is built from the formula:

			\begin{equation}
				\quaternion = 0\vec{i} + \sin(\theta/2)\vec{j} + 0\vec{k} +
					\cos(\theta/2)
			\end{equation}

			\label{item:sample_strategy:quaternion}

			%This produces a rotation of angle $\theta$ around the $y$
			%axis.\todo{explain that y axis is pointing up}

		\item

			The pose $\pose = (\transvec, \quaternion)$ is returned.

	\end{enumerate}

	\subsection{Justification of Sampling Strategy}%
	\label{sec:justification_of_sampling_strategy}

		\glspl{cdpr} have a higher capacity margin towards the centre of
		their workspace\todo{cite something here}. This led to the decision
		to use a normal distribution to sample translations as described in
		Item~\ref{item:sample_strategy:translation} above. In doing so,
		the sampling algorithm has a higher probability to choose
		translations closer to the centre of the workspace, thereby
		increasing the average stability of the trajectory found.

		Items~\ref{item:sample_strategy:angle}
		and~\ref{item:sample_strategy:quaternion} in the list above generate
		a rotation $\theta$ around the $y$-axis. The reason that rotations
		are only drawn around the $y$ axis is two-fold. The first reason is
		that experimentation has shown that the architecture performs faster
		when only drawing rotations around a single axis\todo{back this up
		with numbers}. The second reason is that, by only drawing rotations
		around the $y$ axis, the robot is guaranteed to be upright during
		its trajectory. This produces simpler trajectories, as generating
		completely random quaternions would have the robot change its
		orientation unpredictably.

		The mean and standard deviation as defined in
		Item~\ref{item:sample_strategy:angle} are chosen such that 99.7\% of
		the sampled rotations lie within the range $\theta \in [0, \pi]$.
		The effect of using a uniform distribution within this range instead
		of the distribution of Item~\ref{item:sample_strategy:angle} was
		also investigated. This, however, was deemed an inefficient
		approach, as the end-effector was found to have a tendency to
		unnecessarily swing back and forth around the $y$-axis as it
		progressed through its trajectory. By instead biassing the
		end-effector's orientation towards a certain median position, the
		need for a post-processing operation to simplify the orientation's
		progression was eliminated.

		Following this approach, given an arbitrary start and end pose, the
		architecture finds a path through several poses in such a way that,
		while moving from the start pose to the first sampled pose, the
		robot is turned upright. Then, while in the middle of its
		trajectory, the robot stays upright. Finally, as the robot
		approaches its final pose, it performs any required arbitrary
		rotations to reach the desired goal pose. This is shown
		schematically in Figure~\ref{fig:pose_sampling}.

		\begin{figure}[hb]
			\centering
			\def\svgwidth{\columnwidth}
			\import{res/img/}{pose_sampling.pdf_tex}
			\caption{Pose Sampling}%
			\label{fig:pose_sampling}
		\end{figure}


