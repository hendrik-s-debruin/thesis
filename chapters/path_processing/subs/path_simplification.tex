\section{Path Simplification}%
\label{sec:path_simplification}

	A schematic example of the effect on $\pathsym$ of superfluous poses in
	$\setofposes$ can be seen in the left-hand side of
	Figure~\ref{fig:superfluous_poses}. The algorithms discussed here
	transform such a path into something more closely related to the
	right-hand side of the figure.

	Algorithms were developed to:

	\begin{enumerate}

		\item

			Remove unnecessary poses.

		\item

			Remove corners from the path.

		\item

			Reduce the total amount of rotation along the path.

	\end{enumerate}

	\begin{figure}[hb]
		\centering
		\def\svgwidth{\columnwidth}
		\import{res/img/}{superfluous_poses.pdf_tex}
		\caption{Superfluous Pose Removal}%
		\label{fig:superfluous_poses}
	\end{figure}

	\subsection{Pose Removal}%
	\label{sec:pose_removal}

		A simple recursive algorithm was developed to remove poses from
		$\setofposes$. The pseudocode is given in
		Algorithm~\ref{alg:pose_removal}.

		\begin{algorithm}[ht]
			\caption{Pose Removal}
			\label{alg:pose_removal}
			\begin{algorithmic}[1]
				\Procedure{Remove\_Poses}{}
					\State{}\code{simplified = false}
					\For{$\indexi \in [0, |\setofposes| -2]$}
						\If{$\dist(\pose_{\indexi}, \pose_{\indexi+2}) >
						\dist(\pose_{\indexi}, \pose_{\indexi+1})$}%
						\label{alg:pose_removal:distance_check}
							\If{$\code{Farthest\_Collision\_Free\_Point}(\pose_{\indexi},
							\pose_{\indexi+2}) = \pose_{\indexi+2}$}
								\State{}Remove $\pose_{\indexi+1}$ from
								$\setofposes$
								\State{}$\code{simplifed = true}$
							\EndIf{}
						\EndIf{}
					\EndFor{}
					\If{\code{simplified}}
						\State{}\code{Remove\_Poses}
					\EndIf{}
				\EndProcedure{}
			\end{algorithmic}
		\end{algorithm}

		This algorithm simply removes an intermediate pose if the pose
		before and after it in the sequence in $\setofposes$ can be
		connected by a straight line. However, since a Euclidean distance
		measure (Equation~\ref{eq:euclidean_distance}) is not used, a path
		may actually be \textit{shorter} if an additional intermediate point
		is inserted between two points. For this reason, the algorithm only
		removes points if this would reduce the total travelled distance
		of the path. This is achieved by using the distance measure
		(Equation~\ref{eq:distance_measure_capcity_margin}) in
		Line~\ref{alg:pose_removal:distance_check}.

	\subsection{Corner Removal}%
	\label{sec:corner_removal}

		In some instances, a path cannot be shortened by removing a pose
		from $\setofposes$, since the resulting path would be in collision.
		In such cases, the path may often contain a corner that may be far
		away from the rest of the path. Figure~\ref{fig:path_corner} shows
		a schematic example of this case. To deal with this, an algorithm
		was developed to remove such corners.

		\begin{figure}[hb]
			\centering
			\def\svgwidth{\columnwidth}
			\import{res/img/}{corner_removal.pdf_tex}
			\caption{Path Corner}
			\label{fig:path_corner}
		\end{figure}

		The algorithm tries to find the points $\pose_a, \pose_b$ that
		minimises the total distance travelled along the segment:

		\begin{equation}
			(\pose_a, \pose_b) = \argmin
				(
					\dist(\pose_1, \pose_a) +
					\dist(\pose_a, \pose_b) +
					\dist(\pose_b, \pose_3)
				)
		\end{equation}

		Subject to the constraint:

		\begin{equation}
			\forall
				\obstacle
			\forall
			(
				\pose\in
				\convexhull(\pose_1, \pose_a) \cup \convexhull(\pose_a,
				\pose_b) \cup \convexhull(\pose_b, \pose_3)
			)
			\quad\robot(\pose) \cap \obstacle = \emptyset
		\end{equation}

		This is done by writing the convex hull between two poses as a
		parametric line:

		\begin{align}
			\linevec_1 &= \pose_1 + \timenorm_a(\pose_2 - \pose_1)\\
			\linevec_2 &= \pose_3 + \timenorm_b(\pose_2 - \pose_3)\\
		\end{align}

		Now, using a simple linear search on both $\timenorm_a$ and
		$\timenorm_b$, find the minimum values, $\timenorm_a',
		\timenorm_b'$, for which:

		\begin{align}
			\robot(\convexhull(\pose_1, \linevec_2(\timenorm_b))) \cap
				\obstacle = \emptyset\\
			\robot(\convexhull(\pose_3, \linevec_1(\timenorm_a))) \cap
				\obstacle = \emptyset
		\end{align}

		Now, a two-dimensional binary search is performed on $\timenorm_a$
		and $\timenorm_b$ in the ranges:

		\begin{align}
			\timenorm_a \in [\timenorm_a', 1]\\
			\timenorm_b \in [\timenorm_b', 1]
		\end{align}

		Which produce the poses $\pose_a$ and $\pose_b$ which minimises the
		distance expression and satisfies the constraints of the problem.

	\subsection{Rotation Optimisation}%
	\label{sec:rotation_optimisation}

		Due to the rotation sampling strategy employed as discussed in
		Section~\ref{sec:sample_strategy}, the orientations of the end-effector
		throughout its trajectory will be upright and tend to avoid
		unnecessarily large rotations. However, there may still be some residual
		oscillations around the $y$ axis during its motion. For this reason,
		it is worthwhile to employ a rotation optimisation step.

		The rotation optimisation algorithm uses quaternion
		\gls{slerp}~\cite{bib:math:a_compact_differential_formula_for_the_first_derivative_of_a_unit_quaternion_curve}%
		\cite{bib:math:a_general_construction_scheme_for_unit_quaternion_curves_with_simple_high_order_derivatives}.
		It is summarised in Algorithm~\ref{alg:rotation_optimisation}.

		\begin{algorithm}[ht]
			\caption{Rotation Optimisation}
			\label{alg:rotation_optimisation}
			\begin{algorithmic}[1]
				\Procedure{Optimise\_Rotations}{}
					\ForAll{$%
						\pose_{\indexi} \in
							\setofposes\setminus
								(\pose_{\initial}, \pose_{\goal})
					$}
						\State{}%
							$
								\pose_{\indexi}' \gets
									\code{slerp\_interpolate}
									(
										0.5,
										\pose_{\indexi - 1},
										\pose_{\indexi},
										\pose_{\indexi + 1}
									)
							$
							\label{alg:rotation_optimisation:slerp}
						\If{$
							\dist(\pose_{\indexi - 1}, \pose_{\indexi}') +
							\dist(\pose_{\indexi}', \pose_{\indexi + 1})
							\leq
							\dist(\pose_{\indexi - 1}, \pose_{\indexi}) +
							\dist(\pose_{\indexi}, \pose_{\indexi + 1})
						$}
							\If{$
								\code{path\_okay}
									(
										\pose_{\indexi - 1},
										\pose_{\indexi}',
										\pose_{\indexi + 1}
									)
							$}
								\State{}%
									$
										\pose_{\indexi} \gets \pose_{\indexi}'
									$
							\EndIf{}
						\EndIf{}
					\EndFor{}
				\EndProcedure{}
			\end{algorithmic}
		\end{algorithm}

		The algorithm simply iterates over all the poses in the set and attempts
		to adjust the rotation of each pose so that it is between the rotations
		of the poses that precede and follow it. This is done by
		Line~\ref{alg:rotation_optimisation:slerp}, which returns a pose
		$\pose_{\indexi}'$ that has the same translation component as
		$\pose_{\indexi}$, but whose quaternion is the midway point of the
		\gls{slerp} of poses $\pose_{\indexi - 1}$ and $\pose_{\indexi + 1}$.

		The rotation optimisation algorithm could be run multiple times on
		$\setofposes$ to obtain a progressively more linear rotation progression
		from $\pose_{\initial}$ to $\pose_{\goal}$. However, due to the other
		simplification algorithms, the number of poses in $\setofposes$ tends to
		be small and repeated execution is usually not necessary. Furthermore,
		due to the necessity to check collisions, repeatedly running
		Algorithm~\ref{alg:rotation_optimisation} does get expensive quickly.
		Finally, the sampling strategy discussed in Chapter~\ref{chap:sampling}
		already tends to minimise rotation. This serves to minimise the need to
		run this rotation optimisation procedure multiple times.

	\subsection{Increasing the Capacity Margin}%
	\label{sec:increasing_the_capacity_margin}

		At this point a set $\setofposes$ has been found that is guaranteed to
		be collision-free. However, due to the random nature of the path search
		algorithm, little is done to optimise the overall capacity margin
		$\capacitymargin$ of the trajectory. Two approaches to improve the
		capacity margin are proposed:

		\begin{enumerate}

			\item
				Change the condition for the collision detection algorithm.
				\label{option:change_collision_condition}

			\item
				Post-process $\setofposes$ to improve $\capacitymargin$.
				\label{option:post_process_set_of_poses}

		\end{enumerate}

		Option~\ref{option:change_collision_condition} is to define some minimum
		threshold capacity margin ${\capacitymargin}_{\tol}$ and modify the
		capacity margin submodule ${\logicalpredicate}_{\capacitymargin}$ of the
		collision detection algorithm as follows:

		\begin{equation}
			{\logicalpredicate}_{\capacitymargin} =
				\begin{cases}
					\false, \quad \capacitymargin > {\capacitymargin}_{\tol} \\
					\true, \quad \text{otherwise}
				\end{cases}
		\end{equation}

		This will guarantee that at each point in the trajectory the capacity
		margin is above this minimum limit.

		Option~\ref{option:post_process_set_of_poses} requires that each pose be
		changed according to:

		\begin{equation}
			\forall
				\left(
					{\pose}_{\indexi} \neq {\pose}_{\initial}, {\pose}_{\goal}
				\right)
			\in
				\setofposes,
			%
			{\pose}_{\indexi}' \gets
				{\pose}_{\indexi} + {\gain}_{\indexi}\nabla\capacitymargin({\pose}_{\indexi})
			\label{eq:capacity_margin_increase}
		\end{equation}

		Here, the gain ${\gain}_{\indexi}$ determines how far to move
		${\pose}_{\indexi}$ in the direction of increasing $\capacitymargin$. Of
		course, this is subject to the constraints:

		\begin{equation}
			\dist_{\pathsym}
				\left(
					{\pose}_{\indexi - 1},
					{\pose}_{\indexi}'
				\right)
			+
			\dist_{\pathsym}
				\left(
					{\pose}_{\indexi}',
					{\pose}_{\indexi + 1}
				\right)
			<
			\dist_{\pathsym}
				\left(
					{\pose}_{\indexi - 1},
					{\pose}_{\indexi}
				\right)
			+
			\dist_{\pathsym}
				\left(
					{\pose}_{\indexi},
					{\pose}_{\indexi + 1}
				\right)
			\label{eq:constraint_shorter_distance}
		\end{equation}

		and

		\begin{equation}
			\logicalpredicate(\pathsym') = \false
			\label{eq:constraint_no_collision}
		\end{equation}

		These constraints essentially require that the distance found be shorter
		due to an increase in capacity margin, but still be free of collisions.

		Due to the costs involved in evaluating
		Constraint~\ref{eq:constraint_shorter_distance}, the search on
		${\gain}_{\indexi}$ should be limited. As such, it is beneficial to
		define a discrete set ${{\gain}_{\indexi}}_1 \ldots
		{{\gain}_{\indexi}}_n$ for which
		Equation~\ref{eq:capacity_margin_increase} is called.  If this set of
		gains is kept small, the procedure can be run multiple times to search
		for a path with a higher capacity margin. Of course, this has a
		trade-off with the overall calculation time required.
