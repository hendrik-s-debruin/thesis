\section{Smooth Interpolation}%
\label{sec:smooth_interpolation}

	After the path has been simplified using the algorithms described in
	Section~\ref{sec:path_simplification}, a path consisting of straight lines
	with hard corners is obtained. The next step is to generate a smoother
	trajectory that approximates the path while still avoiding obstacles.
	Figure~\ref{fig:path_smoothing} shows a schematic representation of the
	desired output.

	\begin{figure}[hb]
		\centering
		\def\svgwidth{\columnwidth}
		\import{res/img/}{smoothed_path.pdf_tex}
		\caption{Path Smoothing}%
		\label{fig:path_smoothing}
	\end{figure}

	B-spline curves were used to obtain the desired trajectory. To achieve this,
	each pose $\pose$ in the ordered set of poses $\setofposes$ in the
	simplified path is considered as a control point of the final B-spline
	trajectory. The trajectory is then generated using the well-known B-spline
	recursive relation repeated here for
	completeness~\cite{bib:math:spline_notes}:

	\begin{align}
		\begin{split}
			\pathsym(\timesym) &=
				\sum^{|\setofposes|}_{i=0} N_{i, k}(\timesym)\pose_{\indexi}\\
			%%%%
			N_{\indexi, \indexj}(\timesym) &=
				\frac
				{%
					\timesym - \timesym_{\indexi}
				}
				{%
					\timesym_{\indexi + \indexj} - \timesym_{\indexi}
				}
				N_{\indexi, \indexj - 1}
				+
				\frac
				{%
					\timesym_{\indexi + \indexj + 1} - \timesym
				}
				{%
					\timesym_{\indexi + \indexj + 1} - \timesym_{\indexi + 1}
				}
				N_{\indexi + 1, \indexj - 1}\\
			%%%%%
			N_{\indexi, 0}(\timesym) &=
				\begin{cases}
					1, \quad \timesym_{\indexi} \leq \timesym < \timesym_{\indexi + 1}\\
					0, \quad \text{otherwise}
				\end{cases}\\
			%%%%%
			\vec{\timesym} &=
				\left[
					\begin{matrix}
						\timesym_0, \ldots, \timesym_{k + |\setofposes| + 1}
					\end{matrix}
				\right]
		\end{split}
		\label{eq:b_spline_recursion}
	\end{align}

	Where $k$ is the degree of the B-spline used to approximate the points in
	$\setofposes$ and $\vec{\timesym}$ is the equidistant knot vector that
	produces the uniform spline.

	\subsection{Guaranteeing Smooth Path Safety}%
	\label{sec:guaranteeing_smooth_path_safety}

		Blindly applying Equation~\ref{eq:b_spline_recursion} makes no guarantee
		that the smooth path will be collision free. The generated spline curve
		must therefore be checked for collisions and corrective action must be
		taken in the case that a new collision is detected.

		One technique that is commonly used with B-spline curves is that of knot
		multiplicity. The elements in the knot vector $\vec{\timesym}$ are
		generally spaced equidistantly~\cite{bib:math:spline_notes}. However,
		the multiplicity of a knot may be increased by letting consecutive knots
		assume the same value. This has the effect of drawing the spline closer
		to a control point, but also drops a degree of differentiability at this
		point. A schematic representation of this effect can be seen in
		Figure~\ref{fig:knot_multiplicity}.

		\begin{figure}[hb]
			\centering
			\def\svgwidth{\columnwidth}
			\import{res/img/}{knot_multiplicity.pdf_tex}
			\caption{Increased Knot Multiplicity}%
			\label{fig:knot_multiplicity}
		\end{figure}

		While such an approach works, it presents some drawbacks:

		\begin{itemize}

			\item Loss of smoothness.

				A goal of this thesis is to generate smooth trajectories.
				Dropping a degree of differentiability can create
				discontinuities in the jerk, acceleration or velocity profiles
				of the final trajectory. In the worst case, if a point in the
				path degenerates to $\contdeggeom{0}$ --- that is, a hard corner
				in the position profile --- the end-effector would have to come
				to a complete stop at this point and then accelerate from rest
				again. Stopping the motion in the middle of the trajectory is a
				non-ideal approach and should be avoided as much as possible.

			\item Slow Computation Time.

				In general, higher-degree B-spline curves may be used to
				generate a path in configuration space. If part of the curve is
				in collision, then the multiplicity of the point must be
				increased and the path must be checked again. The path can
				potentially still be in collision and, consequently, this
				process can repeat until the offending point degenerates to
				$\contdeggeom{0}$.

		\end{itemize}

		Instead, this thesis proposes to augment $\setofposes$ with additional
		poses such that the spline curve generated from
		Equation~\ref{eq:b_spline_recursion} is guaranteed to be free of
		collisions. This is achieved by noticing from
		Equation~\ref{eq:b_spline_recursion} that any point in $\pathsym$ is
		obtained from convex combinations of subsequent poses. This means that
		the path is guaranteed to go through the convex hull of the currently
		active control points. The consequence of this is that, if the hull of
		the points can be shown to be collision-free, then the path is
		guaranteed to be collision free automatically. The algorithm developed
		is reported in Algorithm~\ref{alg:set_of_poses_augmentation}.

		\begin{algorithm}[ht]
			\caption{$\setofposes$ Augmentation}%
			\label{alg:set_of_poses_augmentation}
			\begin{algorithmic}[1]
				\Procedure{Augment\_Pose\_Set}{\setofposes{}}
					\State{}Declare new set $\setofposes'$
					\State{}Add first pose in $\setofposes$ to $\setofposes'$
					\ForAll{$\pose_{\indexi} \in \setofposes\setminus
					\pose_{\initial}, \pose_{\goal}$}
						\State{}$\pose_0 = \pose_{\indexi - 1}$
						\State{}$\pose_1 = \pose_{\indexi}$
						\State{}$\pose_2 = \pose_{\indexi + 1}$
						\State{}$\timenorm_1 = 0$
						\State{}$\timenorm_2 = 1$
						\While{$\timenorm_1 \leq 0.5$ and $\timenorm_2 \geq 0.5$}
							\State{}$\pose_a = \pose_0 + \timenorm_1(\pose_1 -
								\pose_0)$
							\State{}$\pose_b = \pose_1 + \timenorm_2(\pose_2 -
								\pose_1)$
							\State{}$\timenorm_1 = \timenorm_1 + \Delta\timenorm$
							\State{}$\timenorm_2 = \timenorm_2 - \Delta\timenorm$
							\If{Farthest\_Collision\_Free\_Point($\pose_a, \pose_b)= \pose_b$}
								\State{}break
							\EndIf{}
						\EndWhile
						\State{}$\setofposes'\code{.push}(\pose_a, \pose_1, \pose_b)$
					\EndFor{}
					\State{}Add last point in $\setofposes$ to $\setofposes'$
					\State{}return $\setofposes'$
				\EndProcedure
			\end{algorithmic}
		\end{algorithm}

		A graphical intuition of Algorithm~\ref{alg:set_of_poses_augmentation}
		can be seen in Figure~\ref{fig:set_of_poses_augmentation}. Essentially,
		it tries to find a shorter straight-line path
		$\vecline{\pose_a}{\pose_b}$ that connects poses $\pose_0$ and
		$\pose_2$, but cuts out $\pose_1$. The search is performed by moving
		$\pose_a$ from $\pose_0$ to the midpoint of the line
		$\vecline{\pose_0}{\pose_1}$ and $\pose_b$ from the midpoint of
		$\vecline{\pose_1}{\pose_2}$ to $\pose_1$.

		\begin{figure}[hbt]
			\centering
			\def\svgwidth{0.7\columnwidth}
			\import{res/img/}{subdivide_control_points.pdf_tex}
			\caption{$\setofposes$ Augmentation}%
			\label{fig:set_of_poses_augmentation}
		\end{figure}

		A shorter path could potentially be found by varying $\timenorm_1$ and
		$\timenorm_2$ in nested loops. However, the call to
		\code{Farthest\_Collision\_Free\_Point} is quite expensive. Nested loops
		would mean that the function is called
		\(
			O
			\left(
				\ceil*
				{%
					{%
						\left(
							\frac
							{%
								1
							}
							{%
								2\Delta\timenorm
							}
						\right)
					}^2
				}
			\right)
		\)
		times, instead of just
		\(
			O
			\left(
				\ceil*
				{%
					\left(
						\frac
						{%
							1
						}
						{%
							2\Delta\timenorm
						}
					\right)
				}
			\right)
		\)
		times. Note here that we require $\Delta\timenorm \in (0, 1]$.
		Therefore, the algorithm in its presented state represents a tradeoff
		between speed and path-length.

		A B-spline curve created using Equation~\ref{eq:b_spline_recursion} on
		the augmented set $\setofposes'$ generated from
		Algorithm~\ref{alg:set_of_poses_augmentation} can be viewed as a set of
		straight lines smoothly interpolated with polynomial bends. The
		algorithm has the following benefits:

		\begin{itemize}

			\item Fast

				The algorithm is linear in the amount of times that the
				collision algorithm needs to be called.

			\item Smooth

				The algorithm ensures that sampled points are connected with the
				maximum degree of smoothness that can be achieved by the degree
				chosen for the B-spline curve.
		\end{itemize}

		A schematic representation of the effect of augmenting $\setofposes$ can
		be seen in Figure~\ref{fig:set_of_poses_augmentation_path}. Note that
		although the figure is schematic in nature, it is drawn using B-spline
		curves and control points as discussed in this section.  As can be seen
		in the figure, adding additional control poses $\pose_a$ and $\pose_b$,
		a collision with $\configurationspace_{\obstacleregion}$ can be avoided
		while still maintaining the smoothness of the curve.

		\begin{figure}[hb]
			\centering
			\def\svgwidth{\columnwidth}
			\import{res/img/}{augmented_control_points.pdf_tex}
			\caption{Effecto of $\setofposes$ Augmentation on $\pathsym$}
			\label{fig:set_of_poses_augmentation_path}
		\end{figure}
