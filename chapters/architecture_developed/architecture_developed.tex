\chapter{Architecture Developed}%
\label{chap:architecture_developed}

	The current chapter gives a brief overview of the architectural design of
	the software.

	\section{Configuration}

		The software is flexible in the sense that it can be used to generate
		plans for a wide range \glspl{cdpr}. This is achieved by defining
		several configuration files that specify the robot and scene topology.

		In the configuration files, poses are written in the following format:

		\begin{equation*}
			x \quad y \quad z \quad qx \quad qy \quad qz \quad w
		\end{equation*}

		Here, $x, y$ and $z$ are the translation components of a pose. $qx, qy$
		and $qz$ are the imaginary components of the orientation quaternion,
		while $w$ is the real part of the quaternion.

		Some configuration files require only coordinates, and not full poses.
		Such configurations are written:

		\begin{equation*}
			x \quad y \quad z
		\end{equation*}

		Configuration files exist for the following components:

		\begin{itemize}

			\item

				The start and goal poses of the end-effector

			\item

				The obstacles in the workspace

			\item

				The shape of the end-effector

			\item

				The shape of the workspace

			\item

				The anchors of the cables

		\end{itemize}

		\subsection{Poses}

			The poses configuration file is merely a list of poses of the form
			defined above.

		\subsection{Obstacles}

			The first line of  the obstacle configuration file defines how many
			obstacles are in the scene. Then, each obstacle is built up of the
			following lines:

			\begin{itemize}

				\item

					A pose configuration describing its position within the
					scene.

				\item

					A list of coordinate lines describing the vertices of the
					obstacle in model space.

			\end{itemize}

		\subsection{End-Effector}

			The end-effector configuration file is merely a list of coordinates
			of the vertices of the end-effector in model space.

		\subsection{Work-Space}

			Similarly to the end-effector, the workspace is also just a list of
			coordinates. These coordinates represent the physical frame of the
			robot.

		\subsection{Cables}

			The cable configuration file lists, for each cable, their proximal
			anchors in the world frame and their distal anchors in the
			end-effector frame on the same line.

	\section{C++ Classes}

		The following section gives a brief overview of the C++ classes
		developed and their function in the overall algorithm.

		\subsection{Polyhedron}

			The \code{ConvexPolyhedron} class is responsible for defining the
			vertices of an object. It also implements the SAT algorithm for
			collision detection.

		\subsection{End-Effector}

			The \code{EndEffector} class inherits from \code{ConvexPolyhedron}
			and adds capabilities for calculating the capacity margin and
			inverse geometric model, while also implementing the distance
			measure used in the architecture.

		\subsection{Cable}

			The \code{Cable} class contains abstractions for testing cable-cable
			collisions, cable-obstacle collisions and cable-end-effector
			collisions.

		\subsection{Sampler}

			The \code{Sampler} class is responsible for sampling poses as
			described in Chapter~\ref{chap:sampling}.

		\subsection{Node}

			The \code{Node} class encodes a vertex of $\topologicalgraph$.
			Connections between nodes are implemented as smart pointers to other
			\class{Node}s.

		\subsection{Spline}

			The \code{Spline} class implements B-Splines using the casteljau
			recursion.

		\subsection{Capacity Margin}

			The capacity margin is not calculated in C++, but instead in Matlab.
			The class \code{CapacityMargin} is a wrapper class that translates
			the Matlab output into a form usable from C++.

		\subsection{Path}

			The \code{Path} class encodes the path as a set of poses. It
			implements the post-processing algorithms as discussed in
			Chapter~\ref{chap:path_processing}.

		\subsection{Trajectory}

			The \code{Trajectory} class is responsible for creating a
			collision-free B-spline interpolation of the points of a
			\code{Path}. It also generates the motion law as described in
			Chapter~\ref{chap:motion_law}.

		\subsection{Rendering}

			The classes \code{Renderer}, \class{Shader}, \class{Camera} and
			\class{Object} encapsulate the code required for 3D rendering with
			OpenGL.

