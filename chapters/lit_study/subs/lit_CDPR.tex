\section{Modelling of Cable-Driven Parallel Robots}%
\label{sec:modelling_of_cable_driven_parallel_robots}

	The main aspects of \gls{cdpr} modelling is discussed in the current
	section. \todo{give better intro/overview of what is discussed here}

	\subsection{\glsentrylong{cdpr} Classification}%
	\label{sec:cdpr_classification}

		\glspl{cdpr} can be classified in different ways. One way to is to
		classify such robots by their degrees of freedom $\robotdof$ and the
		number of cables, $\numcables$. Using these parameters, the degree of
		redundancy, $\degofredundancy$, is defined as:

		\begin{equation}
			\degofredundancy = \numcables - \robotdof
		\end{equation}

		Different classes of \glspl{cdpr} can be obtained by varying the
		relation of $\robotdof$ and $\numcables$.

		\begin{description}

			\item[\glspl{irpm}]

				are under-constrained robots. They cannot withstand any applied
				wrench to the platform. A \gls{cdpr} is an \gls{irpm} when:

				\begin{equation}
					\numcables < \robotdof \leq 6
				\end{equation}

				Robots where $\numcables = \robotdof$ may also be considered
				\glspl{irpm}, as they rely on gravity to impose a force
				equilibrium.

			\item[\glspl{crpm}]

				are \glspl{cdpr} where poses are supported completely by
				tensions in the cables. \glspl{cdpr} are \glspl{crpm} when:

				\begin{equation}
					\numcables = \robotdof + 1
				\end{equation}

			\item[\glspl{rrpm}]

				are robots that have redundant actuation. That is, when:

				\begin{equation}
					\numcables > \robotdof + 1
				\end{equation}
		\end{description}

		\glspl{cdpr} can also be classified according to the type of motion that
		they can generate at the end-effector. These motion types are generally
		classified according to the pattern $x$R$y$T, where $x$ is the degrees
		of rotational freedom and $y$ is the degrees of translational freedom. A
		summary of all possible motion profiles is reported in
		Table~\ref{tab:types_of_cdpr_motion_profiles}.

		\begin{table}[ht]
			\centering
			\begin{tabular}{c c}
				\toprule
				planar robots & spatial robots \\
				\midrule
				1T 			& 	3T \\
				2T 			& 	2R3T \\
				1R2T 		& 	3R3T
			\end{tabular}
			\caption{Types of \gls{cdpr} motion profiles}%
			\label{tab:types_of_cdpr_motion_profiles}
		\end{table}

	\subsection{Geometric Model}%
	\label{sec:geometric_model}

		\glspl{cdpr} differ from classical parallel mechanisms in that they have
		the additional constraint that the cables have to be in tension. The
		ideal model of a \gls{cdpr} assumes that cables are straight lines. That
		is, effects such as sagging are disregarded.

		The geometry of a \gls{cdpr} is reported in
		Figure~\ref{fig:geometry_of_cdpr}.

		\begin{figure}[hb]
			\missingfigure{geometry of CDPR}
			\caption{Geometry of \gls{cdpr}}%
			\label{fig:geometry_of_cdpr}
		\end{figure}


%	\subsection{\glsentrylong{dgm}}
%
%	\subsection{\glsentrylong{igm}}
%
%	\subsection{\glsentrylong{dkm}}
%
%	\subsection{\glsentrylong{ikm}}
%
%	\subsection{\glsentrylong{ddm}}
%
%	\subsection{\glsentrylong{idm}}
%
%	\subsection{Workspace Evaluation}
%
%	\subsection{Cable-Cable Collisions}
%
%	\subsection{Cable-Platform Collisions}
