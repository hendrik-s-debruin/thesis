\section{Modelling of Cable-Driven Parallel Robots}%
\label{sec:modelling_of_cable_driven_parallel_robots}

	The main aspects  of  \gls{cdpr}  modelling  is  discussed	in	the  current
	section.  \todo{give  better  intro/overview  of  what	is	discussed  here}

    \subsection{\glsentrylong{cdpr} Classification}%
    \label{sec:cdpr_classification}

		\glspl{cdpr} can be classified in different ways.	One  way  to  is  to
		classify such robots by their degrees of  freedom  $\robotdof$	and  the
		number of cables, $\numcables$.  Using these parameters, the  degree  of
        redundancy, $\degofredundancy$, is defined as:

        \begin{equation}
            \degofredundancy = \numcables - \robotdof
        \end{equation}

		Different classes  of  \glspl{cdpr}  can  be  obtained	by	varying  the
        relation of $\robotdof$ and $\numcables$.

        \begin{description}

            \item[\glspl{irpm}]

				are under-constrained robots.  They cannot withstand any applied
				wrench to the platform.  A \gls{cdpr}  is  an  \gls{irpm}  when:

                \begin{equation}
                    \numcables < \robotdof \leq 6
                \end{equation}

				Robots where $\numcables = \robotdof$  may	also  be  considered
				\glspl{irpm},  as  they  rely  on  gravity	to	impose	a  force
                equilibrium.

            \item[\glspl{crpm}]

				are  \glspl{cdpr}  where  poses  are  supported  completely   by
				tensions in the cables.   \glspl{cdpr}	are  \glspl{crpm}  when:

                \begin{equation}
                    \numcables = \robotdof + 1
                \end{equation}

            \item[\glspl{rrpm}]

				are robots	that  have	redundant  actuation.	That  is,  when:

                \begin{equation}
                    \numcables > \robotdof + 1
                \end{equation}
        \end{description}

		\glspl{cdpr} can also be classified according to the type of motion that
		they can generate at the end-effector.	These motion types are generally
		classified according to the pattern $x$R$y$T, where $x$ is	the  degrees
        of rotational freedom and $y$ is the degrees of translational freedom. A
		summary   of   all	 possible	motion	 profiles	is	  reported	  in
        Table~\ref{tab:types_of_cdpr_motion_profiles}.

        \begin{table}[ht]
            \centering
            \begin{tabular}{c c}
                \toprule
                planar robots & spatial robots \\
                \midrule
                1T          &   3T \\
                2T          &   2R3T \\
                1R2T        &   3R3T
            \end{tabular}
            \caption{Types of \gls{cdpr} motion profiles}%
            \label{tab:types_of_cdpr_motion_profiles}
        \end{table}

    \subsection{Geometric Model}%
    \label{sec:geometric_model}

		\glspl{cdpr} differ from classical parallel mechanisms in that they have
		the additional constraint that the cables have to be  in  tension.	 The
        ideal model of a \gls{cdpr} assumes that cables are straight lines. That
		is, effects such as sagging are disregarded.   Geometric  equations  are
		derived under  the	standard  model.   Figure~\ref{fig:geometry_of_cdpr}
        shows this model.  \todo{this isn't good wording}

        \begin{figure}[hb]
			\centering
			\def\svgwidth{\columnwidth}
			\import{res/img/}{geometry_of_cdpr.pdf_tex}
            %\missingfigure{geometry of CDPR}
            \caption{Geometry of \gls{cdpr}}%
            \label{fig:geometry_of_cdpr}
        \end{figure}

		In	Figure~\ref{fig:geometry_of_cdpr}  the	proximal  anchor  point   is
		denoted $\proximalanchor$, while the distal anchor	is	$\distalanchor$.
        The loop closure constraint for each cable is:

        \begin{equation}
            \project{%
                \proximalanchor_{\indexi}
            }{\world}
            - \transvec
            - \rotmat{\world}{\platform}
                \project{%
                    \distalanchor_{\indexi}
                }{\platform}
            - \cablevec_{\indexi}
            =
            \zerovec
            \label{eq:cdpr_loop_closure}
        \end{equation}

		The cable unit vector is defined to always point towards the base of the
        robot. It is given by the expression:

        \begin{equation}
            \cableuvec_{\indexi} =
                \frac
                {%
                    \cablevec_{\indexi}
                }
                {%
                    \norm{\cablevec_{\indexi}}
                }
        \end{equation}

        \subsubsection{Inverse Geometric Model}%
        \label{sec:inverse_geometric_model}

            The \gls{igm} $\invgeometricmodel(\pose)$, takes the
			pose $\pose = (\transvec,  \rotmatbare)$  of  the  end-effector  and
			returns   the	lengths   of   the	 cables   of   the	 \gls{cdpr},
            $\cablelengths$. It is found by solving for
            Equation~\ref{eq:cdpr_loop_closure}:

            \begin{equation}
                \cablelengths
                    = \left[
                        \begin{matrix}
                            \norm{\cablevec_1} \\
                            \vdots \\
                            \norm{\cablevec_{\numcables}}
                        \end{matrix}
                    \right]
                    = \invgeometricmodel(\pose)
                    = \left[
                        \begin{matrix}
                            \norm
                            {%
                                \project{\proximalanchor_1}{\world}
                                - \transvec
                                - \rotmat{\world}{\platform}\project{\distalanchor_1}{\platform}
                            }
                            \\
                            \vdots
                            \\
                            \norm
                            {%
                                \project{\proximalanchor_{\numcables}}{\world}
                                - \transvec
                                - \rotmat{\world}{\platform}\project{\distalanchor_{\numcables}}{\platform}
                            }
                        \end{matrix}
                    \right]
                \label{eq:cdpr_inverse_geometric_model}
            \end{equation}

			Even though $\invgeometricmodel(\pose)$ always exists, it  makes  no
			guarantee about whether a given pose is  in  wrench  closure.	This
            needs to be checked separately.

        \subsubsection{Direct Geometric Model}%
        \label{sec:direct_geometric_model}

            The \gls{dgm},
            \(
                \geometricmodel(\pose):\Re^{\numcables}\mapsto\specialEuclideanGroup{3}
            \),
			is	 not   as	simple	 to   derive	as	  the	 \gls{igm}	  in
            Equation~\ref{eq:cdpr_inverse_geometric_model}. In general,
            $\geometricmodel$ may have multiple solutions.

			The calculation of $\geometricmodel$ changes based on  the	topology
			of the \gls{cdpr}.	In	general,  different  algorithms  exist	for:

            \begin{itemize}

                \item[]

                    \glspl{irpm} with $\robotdof > \numcables$

                \item[]

                    \glspl{irpm} with $\robotdof = \numcables$

                \item[]

                    \glspl{crpm} and \glspl{rrpm}

                \item[]

                    Point motion types 2T and 3T

                \item[]

                    Body motion types 1R2T and 3R3T

            \end{itemize}

			Since the \gls{dgm} is generally used in feedback laws for the
			controller~\cite[][page
			121]{bib:cdpr:cable_driven_parallel_robots_theory_and_application}
			and is not directly required in path planning, an overview of the
			algorithms for calculating $\geometricmodel$ is outside the scope of
			this literature study. The algorithms may be found in~\cite[][page
			137]{bib:cdpr:cable_driven_parallel_robots_theory_and_application}.

    \subsection{Kinematic Model}%
    \label{sec:kinematic_model}

		Supposing the  \gls{dgm}  $\pose  =  \geometricmodel(\cablelengths)$  is
		known,	 the   \gls{dkm}   $\kinematicmodel$	can    be	 found	  by
        derivation:

        \begin{equation}
            \dot{\pose}
                = {\kinematicmodel}(\cablelengths, \dot{\cablelengths})
				%=	 \dot{\geometricmodel}(\cablelengths,	\dot{\cablelengths})
                =   \frac
                    {%
                        \partial\geometricmodel(\cablelengths)
                    }
                    {%
                        \partial\timesym
                    }
                =   \frac
                    {%
                        \partial\geometricmodel(\cablelengths)
                    }
                    {%
                        \partial\cablelengths
                    }
                    \frac
                    {%
                        \partial\cablelengths
                    }
                    {%
                        \partial\timesym
                    }
                =   \geometricjac\dot{\cablelengths}
        \end{equation}

		Similarly,	 the   \gls{ikm}   $\invkinematicmodel$    is	 found	  by
        derivation of Equation~\ref{eq:cdpr_inverse_geometric_model}:

        \begin{equation}
            \dot{\cablelengths}
                =   \invkinematicmodel(\pose, \dot{\pose})
                =   \frac
                    {%
                        \partial\invgeometricmodel(\pose)
                    }
                    {%
                        \partial\timesym
                    }
                =   \frac
                    {%
                        \partial\invgeometricmodel(\pose)
                    }
                    {%
                        \partial\pose
                    }
                    \frac
                    {%
                        \partial\pose
                    }
                    {%
                        \partial\timesym
                    }
                =   \invgeometricjac\dot{\pose}
        \end{equation}

		Higher order kinematic models are found by successive derivations of the
        above expressions.

	\subsection{Dynamic Model}%
	\label{sec:dynamic_model}

		For this section, the robot pose $\pose$ is defined in terms of its
		translation $\transvec$ with respect to the world frame $\worldframe$,
		as well as a quaternion $\quaternion$ as follows:

		\begin{equation}
			\pose =
				\left[
					\begin{matrix}
						\transvec \\
						\quaternion
					\end{matrix}
				\right]
		\end{equation}

		Note that a quaternion may be related to the Ration matrix $\rotmatbare$
		in the previous representation $\pose = (\transvec, \rotmatbare)$ as
		follows:

		\begin{equation}
			\rotmatbare =
				\id + 2\quaternionre\skewmat{\quaternionimg} +
					2\skewmat{\quaternionimg}^T\skewmat{\quaternionimg}
		\end{equation}

		The angular velocity $\avel$ is related to the quaternion $\quaternion$
		through the transformation:

		\begin{equation}
			\avel = \quaternionToAvelTrans\dot{\quaternion}
		\end{equation}

		Where
		\(
			\quaternionToAvelTrans=
				2\left[
					\begin{matrix}
						-\quaternionimg & \quaternionre\id + \skewmat{\quaternionimg}
					\end{matrix}
				\right]
		\),
		$\quaternionre$ is the real part of the quaternion, and
		$\skewmat{\quaternionimg}$ is the skew-symmetric matrix associated with
		the imaginary vector part of the quaternion.

		The dynamic model is then defined
		as~\cite{bib:cdpr:cable_driven_parallel_robots_theory_and_application}:

		\begin{align}
			\begin{split}
				\massmat\transformationmat\ddot{\pose} -
					\dampingmat\transformationmat\dot{\pose} +
					\centripitalCoriolisVec
				&=
				\strucmat\forcemagvec + {\wrench_{\platform}}
				\label{eq:dynamic_model}
				\\
				\quaternion^2 - 1 &= 0
			\end{split}
		\end{align}

		Here, the mass matrix $\massmat$ is defined in terms of the mass of the
		platform $\mass_{\platform}$ and its inertia $\inertiamat_{\platform}$ as:

		\begin{equation}
			\massmat =
				\left[
					\begin{matrix}
						\mass_{\platform}\id	& \zerovec \\
						\zerovec 				& \inertiamat_{\platform}
					\end{matrix}
				\right]
		\end{equation}

		The transformation matrix $\transformationmat$ is:

		\begin{equation}
			\transformationmat =
				\left[
					\begin{matrix}
						\id & \zerovec \\
						\zerovec & \quaternionToAvelTrans
					\end{matrix}
				\right]
		\end{equation}

		The matrix

		\begin{equation}
			\dampingmat =
				\diag(
					\dampingmat_{\mathrm{linear}},
					\dampingmat_{\mathrm{rotational}}
				)
		\end{equation}

		collects the damping terms of the system. These can become significant
		for high speed robots, but may be neglected for slower
		speeds~\cite{bib:cdpr:cable_driven_parallel_robots_theory_and_application}.

		The Coriolis effects are collected in $\centripitalCoriolisVec$, given
		by:

		\begin{equation}
			\centripitalCoriolisVec =
			\left[
				\begin{matrix}
					\zerovec \\
					%
					\inertiamat_{\platform}\dot{\quaternionToAvelTrans}\dot{\quaternionimg} +
					\skewmat{\avel}\inertiamat_{\platform}\avel
				\end{matrix}
			\right]
		\end{equation}

		Note that the above discussion models the dynamics of the platform. The
		cables also exhibit dynamic effects such as elastic deformation, thermal
		elongation, creep and vibration. These effects are more pronounced with
		large scale robots or high speed robots and as such are not discussed
		here. More details on these effects may be found in~\cite[][page
		239]{bib:cdpr:cable_driven_parallel_robots_theory_and_application}.

    \subsection{Static Poses}%
    \label{sec:static_poses}

		When the robot is in static equilibrium, the  following  equations	must
        hold:

        \begin{align}
            \sum_{\indexi = 1}^{\numcables}
                \force_{\indexi} +
            \force_{\platform} &= \zerovec \\
            %
            \sum_{\indexi = 1}^{\numcables}
                \distalanchor_{\indexi} \times \force_{\indexi} +
            \torque_{\platform} &= \zerovec
        \end{align}

        In matrix from, these become:

        \begin{equation}
            \left[
                \begin{matrix}
                    \cableuvec_1 &\cdots &\cableuvec_{\numcables} \\
                    %
                    \distalanchor_1\times\cableuvec_1 &\cdots
                    &\distalanchor_{\numcables}\times\cableuvec_{\numcables}
                \end{matrix}
            \right]
            \left[
                \begin{matrix}
                    \forcemag_1 \\
                    \vdots \\
                    \forcemag_{\numcables}
                \end{matrix}
            \right]
            +
            \left[
                \begin{matrix}
                    \force_{\platform} \\
                    \torque_{\platform}
                \end{matrix}
            \right]
            =
            \zerovec
        \end{equation}

        Which can be condensed to the form:

        \begin{equation}
            \strucmat(\transvec, \rotmatbare)\forcemagvec +
            \wrench_{\platform} = \zerovec
            \label{eq:structure_equation}
        \end{equation}

        This is known as the structure equation.

        \subsubsection{Wrench-Closure Poses}%
        \label{sec:wrench_closure_poses}

			Not every possible pose of the robot can be supported by the  cables
            in static equilibrium. A pose is said to be in wrench-closure if any
			external platform wrench can be balanced by positive forces  in  the
            cables. That is,

            \begin{equation}
                \forall
                    \left(
                        {\wrench}_{\platform}\in\Re^{\robotdof}
                    \right)
                \exists
                    \left(
                        \forcemagvec\in\Re^{\numcables}
                    \right)
                \quad\suchthat\quad
                    \left(
                        \strucmat(\transvec, \rotmatbare)\forcemagvec + \wrench_{\platform} = \zerovec
                    \right)
                    \wedge
                    \left(
                        \forcemagvec > \zerovec
                    \right)
                \label{eq:wrench_closure_constraint}
            \end{equation}

        \subsubsection{Wrench-Feasible Poses}%
        \label{sec:wrench_feasible_poses}

			Wrench feasibility is similar to wrench  closure,  except  it  takes
			into account the minimum and maximum cable forces of the \gls{cdpr}.
			The maximum cable forces ensure that  the  cables  do  not	fail  in
			tension, whereas the minimum cable forces are chosen to ensure	that
			there is no slack in the cables.  A pose $\pose$ is  wrench-feasible
            for a set of wrenches $\setofwrenches$ if:

            \begin{equation}
                \forall
                    \left(
                        {\wrench}_{\platform}\in\setofwrenches
                    \right)
                \exists
                    \left(
                        \forcemagvec\in\Re^{\numcables}
                    \right)
                \suchthat
                    \left(
                        \strucmat(\pose)\forcemagvec + \wrench_{\platform} = \zerovec
                    \right)
                    \wedge
                    \left(
                        0 < \forcemag_{\min} \leq \forcemag_{\indexi} \leq \forcemag_{\max}
                        \quad\forall \indexi \in
                            \left\{
                                1,\ldots,\numcables
                            \right\}
                    \right)
            \end{equation}

			\todo{the following discussion is only relevant to	CRPM  and  RRPM}

            \todo{$\mat{H}$ is not in the nomenclature}

            The set of solutions of the structure
            equation~\ref{eq:structure_equation} is:

            \begin{equation}
                \setsolstruceq =
                    \left\{
                        \forcemagvec =
                            -\pseudoinv{\strucmat}\wrench_{\platform} +
                            \mat{H}\vec{\gain}
                        \,\middle|\,
                            \vec{\gain \in \Re^{\degofredundancy}}
                    \right\}
            \end{equation}

			Where the vectors in $\mat{H}$ form a  basis  of  $\ker{\strucmat}$.

            The set of feasible force distributions is:

            \begin{equation}
                \setoffeasibleforces =
                    \left\{
                        \forcemagvec \in \Re^{\numcables}
                    \,\middle|\,
                        0 < \forcemag_{\min} \leq \forcemag_{\indexi} \leq \forcemag_{\max}
							\quad\forall  \indexi  \in	 \{1,\ldots,\numcables\}
                    \right\}
            \end{equation}

			A  pose   is   therefore   wrench	feasible   if	and   only	 if:

            \begin{equation}
				\setsolstruceq	 \cap	\setoffeasibleforces   \neq    \emptyset
            \end{equation}

			The set  $\setsolstruceq  \cap	\setoffeasibleforces$  is  a  convex
			polyhedron.  That is, if both $\forcemagvec_1$ and	$\forcemagvec_2$
            are feasible, than the force distribution

            \begin{equation}
				\forcemagvec = \gain\forcemagvec_1 + (1 -  \gain)\forcemagvec_2,
                    \quad\forall\gain\in[0;1]
                \label{eq:convexity_of_feasible_forces}
            \end{equation}

            is also a feasible solution

        \subsection{Wrenches Generated by the Robot}%
        \label{sec:wrenches_generated_by_the_robot}

            Similarly to the sets described in
			Section~\ref{sec:wrench_feasible_poses}, the set  of  wrenches	that
			can be generated  by  the  \gls{cdpr}  can	also  be  characterised.
			Analogously to	Equation~\ref{eq:convexity_of_feasible_forces},  the
			set of wrenches $\setofwrenches$ that the robot can  generate  at  a
			pose $\pose$ is a convex polyhedron.  That is, if it is known that a
			robot	can    generate    the	  discrete	  set	 of    wrenches:

            \begin{equation}
                \setofwrenches_{\discrete}(\pose) =
                    \left\{
                        \wrench_1, \ldots, \wrench_{\robotdof}
                    \right\}
            \end{equation}

            Then the full set of wrenches the robot is:

            \begin{equation}
                \setofwrenches(\pose) \supseteq \convexhull(\setofwrenches_{\discrete}(\pose))
            \end{equation}

			To ensure that $\setofwrenches$ can be generated, it  is  sufficient
			to	check  the	$2^{\dim\setofwrenches}$   corners	 of   the	set.
            \todo{cite this section explicitly, p 95}

    \subsection{Singularities of \glsentrylongpl{cdpr}}%
    \label{sec:singularities_of_cdprs}

		The singularities of \glspl{cdpr} may be analysed by collecting the loop
		closure equations~\ref{eq:cdpr_loop_closure}  of  each	cable  into  the
        form:

        \begin{equation}
            \loopclosureconstraints(\pose,\cablelengths) = \zerovec
        \end{equation}

		where $\loopclosureconstraints$ is a simple stacking  of  the  left-hand
		side of Equation~\ref{eq:cdpr_loop_closure}  for  each	cable  $\indexi$
        into a vector.

        Differentiation of this expression leads to:

        \begin{equation}
            \frac
            {%
                \partial\loopclosureconstraints(\pose,\cablelengths)
            }
            {%
                \partial\pose
            }
            \delta\pose
            +
            \frac
            {%
                \partial\loopclosureconstraints(\pose,\cablelengths)
            }
            {%
                \partial\cablelengths
            }
            \delta\cablelengths
            = \jac_A\delta\pose + \jac_B\delta\cablelengths
            = \zerovec
        \end{equation}
        \todo{$\jac_A$ and $\jac_B$ are not in nomenclature}

        Where

        \begin{equation}
            \jac_A =
                -\left[
                    \begin{matrix}
                        2\left(\proximalanchor_1 - \rotmatbare\distalanchor_1 - \transvec\right) &
                        \cdots &
                        2\left(\proximalanchor_{\robotdof} - \rotmatbare\distalanchor_{\robotdof} - \transvec\right) &
                        \\
                        %
                        2\rotmatbare\distalanchor_1 \times
                            \left(
                                \proximalanchor_1 - \rotmatbare\distalanchor_1 - \transvec
                            \right) &
                        \cdots &
                        2\rotmatbare\distalanchor_{\robotdof} \times
                            \left(
                                \proximalanchor_{\robotdof} - \rotmatbare\distalanchor_{\robotdof} - \transvec
                            \right)
                    \end{matrix}
                \right]
        \end{equation}

        and

        \begin{equation}
            \jac_B =
                \left[
                    \begin{matrix}
                        2\cablelength_1 & & \\
                        & \ddots & \\
                        & & 2\cablelength_{\numcables}
                    \end{matrix}
                \right]
        \end{equation}

        The robot loses a degree of freedom if
        \(
            \det\jac_B = 0
		\), that is, if a cable is fully retracted.  The robot loses its ability
        to withstand certain wrenches if
        \(
            \det\jac_A = 0
        \).

		Since $\jac_A$ is not necessarily square,  a  necessary  and  sufficient
		condition to test for its singularity is~\cite[][page
		127]{bib:cdpr:cable_driven_parallel_robots_theory_and_application}:

        \begin{equation}
            \rank \strucmat(\pose) < \robotdof
			\label{eq:cond_singularity}
        \end{equation}

    \subsection{The Workspace of a \glsentrylong{cdpr}}%
    \label{sec:the_workspace_of_a_cdpr}

        Characterising the workspace
		$\workspace\subset\specialEuclideanGroup{3}$   of	a	\gls{cdpr}	  is
        important for its operation. This section gives an overview of different
		commonly used workspace subsets, as well as how  the  workspace  may  be
        calculated and represented.

        \subsubsection{Common Workspace Subsets}%
        \label{sec:common_workspace_subsets}

			A	brief	overview   of	workspace	subsets   is   given   here.
			\todo{these   subscripts	are    not	  in	the    nomenclature}
            \todo{cite p 178}

            \paragraph{Constant Orientation Workspace}%
            \label{sec:constant_orientation_workspace}

				The constant orientation workspace is the set of points that can
				be	 reached   without	 changing	the    robot	orientation:

                \begin{equation}
                    \workspace_{\mathrm{CO}}(\rotmatbare_0) =
                        \left\{
                            \transvec\in\Re^3
                            \,\middle|\,
                            \pose = (\transvec, \rotmatbare_0)
                        \right\}
                \end{equation}

            \paragraph{Orientation Workspace}%
            \label{sec:orientation_workspace}

				The orientation workspace is the rotations that may  be  reached
                at a given position:

                \begin{equation}
                    \workspace_{\mathrm{O}}(\transvec_0) =
                        \left\{
                            \rotmatbare\in\specialOrthonormalGroup{3}
                            \,\middle|\,
                            \pose = (\transvec_0, \rotmatbare)
                        \right\}
                \end{equation}

            \paragraph{Inclusion Orientation Workspace}%
            \label{sec:inclusion_orientation_workspace}

				The inclusion orientation workspace is a generalisation  of  the
				constant orientation workspace that includes a set of rotations,
                $\setofrotationmatrices$:

                \begin{equation}
                    \workspace_{\mathrm{IO}}(\setofrotationmatrices) =
                        \left\{
                            \transvec \in \Re^3
                            \,\middle|\,
                            \pose = (\transvec, \rotmatbare),
                            \rotmatbare\in\setofrotationmatrices
                        \right\}
                \end{equation}

            \paragraph{Maximum Workspace}%
            \label{sec:maximum_workspace}

				The maximum workspace,	$\workspace_{\max}$,  is  the  inclusion
				orientation   workspace    where	$\setofrotationmatrices    =
                \specialOrthonormalGroup{3}$

            \paragraph{Total Orientation Workspace}%
            \label{sec:total_orientation_workspace}

				The    total	orientation    workspace	is	  defined	 by:

                \begin{equation}
                    \workspace_{\mathrm{TO}}(\setofrotationmatrices) =
                        \left\{
                            \transvec \in \Re^3
                            \,\middle|\,
                            \pose = (\transvec, \rotmatbare)
                            \quad\forall\rotmatbare\in\setofrotationmatrices
                        \right\}
                \end{equation}

            \paragraph{Dextrous Workspace}%
            \label{sec:dextroux_workspace}

				The dextrous workspace, $\workspace_{\mathrm{D}}$, is the  total
				orientation   workspace    where	$\setofrotationmatrices    =
                \specialOrthonormalGroup{3}$.

            \paragraph{Workspace Aspect}%
            \label{sec:workspace_aspect}

				Workspaces may be separated into different aspects just as with
				other types of parallel robots. Aspects may be separated by
				singularity surfaces, wrench closure and wrench feasible
				regions. However, computing the number and shape of these
				aspects remains an open
				problem~\cite{bib:cdpr:cable_driven_parallel_robots_theory_and_application}.

		\subsubsection{Workspace Representation}%
		\label{sec:workspace_representation}

			The workspaces of Section~\ref{sec:common_workspace_subsets} need to
			be represented in a manner suitable for computer manipulation. Some
			methods are mentioned
			below~\cite{bib:cdpr:cable_driven_parallel_robots_theory_and_application}\cite{bib:planning:planning_algorithms}.

			\begin{enumerate}

				\item Discrete Samples

					The workspace may be  represented  as  a  simple  set  of
					poses $\setofposes$:

					\begin{equation}
						\setofposes =
							\left\{
								\pose
								\,\middle|\,
								\pose\in\workspace
							\right\}
					\end{equation}

					The workspace may be sampled as a grid, but  more
					sophisticated methods, such as those mentioned in
					Section~\ref{sec:sampling_strategies} may be used.

				\item Solid Geometry

					The workspace may be separated into simplex cells.	This
					forms a continuous representation of the workspace and is
					similar to the cells mentioned in
					Section~\ref{sec:combinatorial_motion_planning}.

				\item Polynomial Form

					The workspace may be modelled with semi-algebraic models
					such as those discussed in
					Section~\ref{sec:semi_algebraic_representation_of_bodies}.

			\end{enumerate}

		\subsubsection{Criteria on Poses in the Workspace}%
		\label{sec:criteria_on_poses_in_the_workspace}

			To ensure that $\pose\in\workspace$, a	candidate  pose  needs	to
			meet certain criteria.  These criteria may  include	the
			following

			\begin{enumerate}

				\item Wrench closure of the pose

				\item Absence of Singularities at the pose

				\item Self-Collision

				\item Obstacle collision

				\item Limits in allowable  angles  of  proximal  and  distal
					cables

				\item Cable length

				%\item Spooling capacity
				\todo{commented out}

			\end{enumerate}

			\paragraph{Wrench-Closure Workspace}%
			\label{sec:wrench_closure_workspace}

				The Wrench-Closure constraint,
				Equation~\ref{eq:wrench_closure_constraint}, can be checked
				using an analysis of the structure matrix $\strucmat$.

				\todo{$\vec{k}$ and $\kappa$ are not in nomenclature}

				Define $\vec{k} \in \ker(\strucmat)$.  A measure for the quality
				of the force distribution of the cables, $\kappa$, can  be
				defined  as follows:

				\begin{equation}
					\kappa  =
						\begin{cases}
							\frac
							{%
								\min{\vec{k}}
							}
							{%
								\max{\vec{k}}
							},
							& \min{\vec{k}} > 0
							\\
							\frac
							{%
								\max{\vec{k}}
							}
							{%
								\min{\vec{k}}
							},
							& \max{\vec{k}} < 0
							\\
							0, & \text{otherwise}
						\end{cases}
				\end{equation}

				If	$\kappa  =	0$,  we  can  deduce  $\pose  \not\in
				\workspace$.  $\kappa \in (0, 1]$ serves as a measure for the
				quality of the force distribution, with $\kappa = 1$ being
				optimal.

			\paragraph{Cable Length}%
			\label{sec:cable_length}

				A constraint on the cable length can be written as:

				\begin{equation}
					\cablelengths_{\min}
						\leq \invgeometricmodel(\transvec, \rotmatbare)
						\leq \cablelengths_{\max}
				\end{equation}

				%This forms a convex polyhedron which may be verified by testing
				its %vertices.

				Such a constraint is straightforward to test.\todo{convex
				polyhedron commented out}

			%\subsubsection{Dynamic Workspace}%
			%\label{sec:dynamic_workspace}
			\todo{section on dynamic workspace should be here}

			\paragraph{Singularities in Workspace}%
			\label{sec:singularities_in_workspace}

				For $\pose\in\workspace$ to hold true, the pose $\pose$ may not
				be in singularity. The condition in
				Equation~\ref{eq:cond_singularity} may be used to test for this.

			\paragraph{Cable-Cable Interference and Cable-Platform Interference}%
			\label{sec:cable_cable_interference_and_cable_platform_interference}

				For a pose to belong to the workspace, the robot may not be in
				self-collision. Most methods that detect such collisions are
				dependent on specific \gls{cad} models. As such, they are not
				collected here, but references may be found in~\cite[][page
				171]{bib:cdpr:cable_driven_parallel_robots_theory_and_application}.

		\subsubsection{Determining the Workspace}%
		\label{sec:determining_the_workspace}

			Analogously to path planning, methods for determining the workspace
			may be subdivided into discrete and continuous methods.

			\paragraph{Discrete Workspace Determination}%
			\label{sec:discrete_workspace_determination}

				Workspaces may be characterised in a discrete fashion. The
				procedure to determine a discrete workspace is the same as the
				discrete representation of
				Section~\ref{sec:workspace_representation}.

			\paragraph{Continuous Workspace Determination}%
			\label{sec:continuous_workspace_determination}

				One method to evaluate the workspace is through the use of
				interval analysis. This method breaks the workspace into
				a finite set intervals within which it can guarantee all poses
				to be inside or outside of the workspace.

				Interval analysis solves the \gls{csp} problem. Constraints such
				as those in Section~\ref{sec:criteria_on_poses_in_the_workspace}
				can be solved with this technique by collecting them and
				rewriting them into the form:

				\begin{equation}
					\cspfunc(\calcvars, \vervars) > \zerovec \quad\forall \vervars\in\verdomain
				\end{equation}

				For the purposes of the workspace analysis, the calculation
				variables are the pose of the robot, that is, $\calcvars =
				\pose$. $\vervars$ collects the verification variables of the
				constraints. The domain of the problem is $\verdomain$. The
				solution set of this \gls{csp} problem, $\cspsolutionset$, is
				the workspace of the robot. That is, $\cspsolutionset =
				\workspace$. In general, the solution set is given by:

				\begin{equation}
					\cspsolutionset =
					\left\{
						\calcvars \in \Re^{\robotdof}
						\,\middle|\,
						\cspfunc(\calcvars, \vervars) > \zerovec
						\quad\forall\vervars\in\verdomain
					\right\}
				\end{equation}

				The set of calculation variables considered forms the search
				space, $\cspsearchspace$. The types of workspaces listed in
				Section~\ref{sec:common_workspace_subsets} may be found by aptly
				choosing $\cspsearchspace$ and $\verdomain$.

				\todo{interval is broken in nomenclature}

				\gls{csp} solvers look for intervals $\interval{\calcvars}$ for
				which there is no change in the truth value of the statement:
				\(
					\cspfunc(\calcvars)>\zerovec
						\quad\forall\calcvars\in\interval{\calcvars}
				\).

				A \gls{csp} solver may deduce the properties of an interval as
				follows:

				\begin{equation}
					\begin{cases}
						\inf(\cspfunc(\interval{\calcvars})) > 0, &
							\cspfunc(\calcvars) > 0
								\quad \forall\calcvars\in\interval{\calcvars}\\
						%
						{\sup(\cspfunc(\interval{\calcvars})}_{\indexi}) < 0, &
							\nexists\calcvars\in\interval{\calcvars}\,\suchthat\,
								\cspfunc(\calcvars) > 0 \\
						%
						\text{Otherwise}, & \text{no conclusion}
					\end{cases}
				\end{equation}

				An algorithm for a simple \gls{csp} solver is reported in
				Algorithm~\ref{alg:interval_csp_solver}~\cite{bib:cdpr:cable_driven_parallel_robots_theory_and_application}.

				\begin{algorithm}[ht]
					\caption{Interval CSP Solver}\label{alg:interval_csp_solver}
					\begin{algorithmic}[1]
						\Procedure{IntervalCSPSolver}{$\cspsearchspace$}

							\State{} Store approximation
								$\{{\interval{\calcvars}}_1, \cdots,
								{\interval{\calcvars}}_n\}$
								of the search space $\cspsearchspace$ in a queue
								$\queue_T$

							\State{} Create a queue $\queue_S$ for the solutions

							\State{} Create a queue $\queue_I$ for invalid
							intervals

							\State{} Create a queue $\queue_U$ for undesirably
							small intervals

							\While{$\queue_T \neq \emptyset$}

								\State{} $\interval{\calcvars} \gets \queue_T\code{.next}$

								\If{$\diam(\interval{\calcvars}) < \threshold$}

									\State{} $\queue_U\code{.add}(\interval{\calcvars})$

								\Else{}
									\State{} $\interval{\vec{h}} \gets \cspfunc(\interval{\calcvars}) $

									\If{$\inf{\interval{\vec{h}}} > \zerovec$}
										\State{} $\queue_S\code{.add}(\interval{\calcvars})$
									\ElsIf{$\exists \indexi \suchthat {\sup\interval{\vec{h}}}_{\indexi} < 0$}
										\State{} $\queue_I\code{.add}(\interval{\calcvars}$)
									\Else{}
										\State{} Split $\interval{\calcvars}$ into boxes
										\(
											\{
												{\interval{\calcvars}}_1,
												\cdots,
												{\interval{\calcvars}}_m
											\}
										\) and put them into $\queue_T$
									\EndIf{}
								\EndIf{}
							\EndWhile{}
						\EndProcedure{}
					\end{algorithmic}
				\end{algorithm}

				\todo{bisecting box section, p 203, could go here, but leaving
				out for now}
%
%   \subsection{\glsentrylong{ddm}}
%
%   \subsection{\glsentrylong{idm}}
%
