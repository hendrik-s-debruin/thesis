\section{Modelling of Cable-Driven Parallel Robots}%
\label{sec:modelling_of_cable_driven_parallel_robots}

	The main aspects of \gls{cdpr} modelling is discussed in the current
	section. \todo{give better intro/overview of what is discussed here}

	\subsection{\glsentrylong{cdpr} Classification}%
	\label{sec:cdpr_classification}

		\glspl{cdpr} can be classified in different ways. One way to is to
		classify such robots by their degrees of freedom $\robotdof$ and the
		number of cables, $\numcables$. Using these parameters, the degree of
		redundancy, $\degofredundancy$, is defined as:

		\begin{equation}
			\degofredundancy = \numcables - \robotdof
		\end{equation}

		Different classes of \glspl{cdpr} can be obtained by varying the
		relation of $\robotdof$ and $\numcables$.

		\begin{description}

			\item[\glspl{irpm}]

				are under-constrained robots. They cannot withstand any applied
				wrench to the platform. A \gls{cdpr} is an \gls{irpm} when:

				\begin{equation}
					\numcables < \robotdof \leq 6
				\end{equation}

				Robots where $\numcables = \robotdof$ may also be considered
				\glspl{irpm}, as they rely on gravity to impose a force
				equilibrium.

			\item[\glspl{crpm}]

				are \glspl{cdpr} where poses are supported completely by
				tensions in the cables. \glspl{cdpr} are \glspl{crpm} when:

				\begin{equation}
					\numcables = \robotdof + 1
				\end{equation}

			\item[\glspl{rrpm}]

				are robots that have redundant actuation. That is, when:

				\begin{equation}
					\numcables > \robotdof + 1
				\end{equation}
		\end{description}

		\glspl{cdpr} can also be classified according to the type of motion that
		they can generate at the end-effector. These motion types are generally
		classified according to the pattern $x$R$y$T, where $x$ is the degrees
		of rotational freedom and $y$ is the degrees of translational freedom. A
		summary of all possible motion profiles is reported in
		Table~\ref{tab:types_of_cdpr_motion_profiles}.

		\begin{table}[ht]
			\centering
			\begin{tabular}{c c}
				\toprule
				planar robots & spatial robots \\
				\midrule
				1T 			& 	3T \\
				2T 			& 	2R3T \\
				1R2T 		& 	3R3T
			\end{tabular}
			\caption{Types of \gls{cdpr} motion profiles}%
			\label{tab:types_of_cdpr_motion_profiles}
		\end{table}

	\subsection{Geometric Model}%
	\label{sec:geometric_model}

		\glspl{cdpr} differ from classical parallel mechanisms in that they have
		the additional constraint that the cables have to be in tension. The
		ideal model of a \gls{cdpr} assumes that cables are straight lines. That
		is, effects such as sagging are disregarded. Geometric equations are
		derived under the standard model.  Figure~\ref{fig:geometry_of_cdpr}
		shows this model.  \todo{this isn't good wording}

		\begin{figure}[hb]
			\missingfigure{geometry of CDPR}
			\caption{Geometry of \gls{cdpr}}%
			\label{fig:geometry_of_cdpr}
		\end{figure}

		In Figure~\ref{fig:geometry_of_cdpr} the proximal anchor point is
		denoted $\proximalanchor$, while the distal anchor is $\distalanchor$.
		The loop closure constraint for each cable is:

		\begin{equation}
			\project{%
				\proximalanchor_{\indexi}
			}{\world}
			- \transvec
			- \rotmat{\world}{\platform}
				\project{%
					\distalanchor_{\indexi}
				}{\platform}
			- \cablevec_{\indexi}
			=
			\zerovec
			\label{eq:cdpr_loop_closure}
		\end{equation}

		The cable unit vector is defined to always point towards the base of the
		robot. It is given by the expression:

		\begin{equation}
			\cableuvec_{\indexi} =
				\frac
				{%
					\cablevec_{\indexi}
				}
				{%
					\norm{\cablevec_{\indexi}}
				}
		\end{equation}

		\subsubsection{Inverse Geometric Model}%
		\label{sec:inverse_geometric_model}

			The Inverse Geometric Model, $\invgeometricmodel(\pose)$, takes the
			pose $\pose = (\transvec, \rotmatbare)$ of the end-effector and
			returns the lengths of the cables of the \gls{cdpr},
			$\cablelengths$. It is found by solving for
			Equation~\ref{eq:cdpr_loop_closure}:

			\begin{equation}
				\cablelengths
					= \left[
						\begin{matrix}
							\norm{\cablevec_1} \\
							\vdots \\
							\norm{\cablevec_{\numcables}}
						\end{matrix}
					\right]
					= \invgeometricmodel(\pose)
					= \left[
						\begin{matrix}
							\norm
							{%
								\project{\proximalanchor_1}{\world}
								- \transvec
								- \rotmat{\world}{\platform}\project{\distalanchor_1}{\platform}
							}
							\\
							\vdots
							\\
							\norm
							{%
								\project{\proximalanchor_{\numcables}}{\world}
								- \transvec
								- \rotmat{\world}{\platform}\project{\distalanchor_{\numcables}}{\platform}
							}
						\end{matrix}
					\right]
				\label{eq:cdpr_inverse_geometric_model}
			\end{equation}

			Even though $\invgeometricmodel(\pose)$ always exists, it makes no
			guarantee about whether a given pose is in wrench closure. This
			needs to be checked separately.

		\subsubsection{Direct Geometric Model}%
		\label{sec:direct_geometric_model}

			The direct geometric model,
			\(
				\geometricmodel(\pose):\Re^{\numcables}\mapsto\specialEuclideanGroup{3}
			\),
			is not as simple to derive as the inverse geometric model in
			Equation~\ref{eq:cdpr_inverse_geometric_model}. In general,
			$\geometricmodel$ may have multiple solutions.

			The calculation of $\geometricmodel$ changes based on the topology
			of the \gls{cdpr}. In general, different algorithms exist for:

			\begin{itemize}

				\item[]

					\glspl{irpm} with $\robotdof > \numcables$

				\item[]

					\glspl{irpm} with $\robotdof = \numcables$

				\item[]

					\glspl{crpm} and \glspl{rrpm}

				\item[]

					Point motion types 2T and 3T

				\item[]

					Body motion types 1R2T and 3R3T

			\end{itemize}

			Since the direct geometric model is generally used in feedback laws
			for the controller \todo{cite page 139} and is not directly
			required in path planning, an overview of the algorithms for
			calculating $\geometricmodel$ is outside the scope of this
			literature study. The algorithms may be found in\todo{cite page
			154}.

	\subsection{Kinematic Model}%
	\label{sec:kinematic_model}

		Supposing the direct geometric model $\pose =
		\geometricmodel(\cablelengths)$ is known, the direct kinematic model
		$\kinematicmodel$ can be found by derivation:

		\begin{equation}
			\dot{\pose}
				= {\kinematicmodel}(\cablelengths, \dot{\cablelengths})
				%= \dot{\geometricmodel}(\cablelengths, \dot{\cablelengths})
				=	\frac
					{%
						\partial\geometricmodel(\cablelengths)
					}
					{%
						\partial\timesym
					}
				=	\frac
					{%
						\partial\geometricmodel(\cablelengths)
					}
					{%
						\partial\cablelengths
					}
					\frac
					{%
						\partial\cablelengths
					}
					{%
						\partial\timesym
					}
				= 	\geometricjac\dot{\cablelengths}
		\end{equation}

		Similarly, the inverse kinematic model $\invkinematicmodel$ is found by
		derivation of Equation~\ref{eq:cdpr_inverse_geometric_model}:

		\begin{equation}
			\dot{\cablelengths}
				= 	\invkinematicmodel(\pose, \dot{\pose})
				= 	\frac
					{%
						\partial\invgeometricmodel(\pose)
					}
					{%
						\partial\timesym
					}
				= 	\frac
					{%
						\partial\invgeometricmodel(\pose)
					}
					{%
						\partial\pose
					}
					\frac
					{%
						\partial\pose
					}
					{%
						\partial\timesym
					}
				= 	\invgeometricjac\dot{\pose}
		\end{equation}

		Higher order kinematic models are found by successive derivations of the
		above expressions.

	\subsection{Static Poses}%
	\label{sec:static_poses}

		When the robot is in static equilibrium, the following equations must
		hold:

		\begin{align}
			\sum_{\indexi = 1}^{\numcables}
				\force_{\indexi} +
			\force_{\platform} &= \zerovec \\
			%
			\sum_{\indexi = 1}^{\numcables}
				\distalanchor_{\indexi} \times \force_{\indexi} +
			\torque_{\platform} &= \zerovec
		\end{align}

		In matrix from, these become:

		\begin{equation}
			\left[
				\begin{matrix}
					\cableuvec_1 &\cdots &\cableuvec_{\numcables} \\
					%
					\distalanchor_1\times\cableuvec_1 &\cdots
					&\distalanchor_{\numcables}\times\cableuvec_{\numcables}
				\end{matrix}
			\right]
			\left[
				\begin{matrix}
					\forcemag_1 \\
					\vdots \\
					\forcemag_{\numcables}
				\end{matrix}
			\right]
			+
			\left[
				\begin{matrix}
					\force_{\platform} \\
					\torque_{\platform}
				\end{matrix}
			\right]
			=
			\zerovec
		\end{equation}

		Which can be condensed to the form:

		\begin{equation}
			\strucmat(\transvec, \rotmatbare)\forcemagvec +
			\wrench_{\platform} = \zerovec
			\label{eq:structure_equation}
		\end{equation}

		This is known as the structure equation.

		\subsubsection{Wrench-Closure Poses}%
		\label{sec:wrench_closure_poses}

			Not every possible pose of the robot can be supported by the cables
			in static equilibrium. A pose is said to be in wrench-closure if any
			external platform wrench can be balanced by positive forces in the
			cables. That is,

			\begin{equation}
				\forall
					\left(
						{\wrench}_{\platform}\in\Re^{\robotdof}
					\right)
				\exists
					\left(
						\forcemagvec\in\Re^{\numcables}
					\right)
				\quad\suchthat\quad
					\left(
						\strucmat(\transvec, \rotmatbare)\forcemagvec + \wrench_{\platform} = \zerovec
					\right)
					\wedge
					\left(
						\forcemagvec > \zerovec
					\right)
			\end{equation}

		\subsubsection{Wrench-Feasible Poses}%
		\label{sec:wrench_feasible_poses}

			Wrench feasibility is similar to wrench closure, except it takes
			into account the minimum and maximum cable forces of the \gls{cdpr}.
			The maximum cable forces ensure that the cables do not fail in
			tension, whereas the minimum cable forces are chosen to ensure that
			there is no slack in the cables. A pose $\pose$ is wrench-feasible
			for a set of wrenches $\setofwrenches$ if:

			\begin{equation}
				\forall
					\left(
						{\wrench}_{\platform}\in\setofwrenches
					\right)
				\exists
					\left(
						\forcemagvec\in\Re^{\numcables}
					\right)
				\suchthat
					\left(
						\strucmat(\pose)\forcemagvec + \wrench_{\platform} = \zerovec
					\right)
					\wedge
					\left(
						0 < \forcemag_{\min} \leq \forcemag_{\indexi} \leq \forcemag_{\max}
						\quad\forall \indexi \in
							\left\{
								1,\ldots,\numcables
							\right\}
					\right)
			\end{equation}

			\todo{the following discussion is only relevant to CRPM and RRPM}

			\todo{$\mat{H}$ is not in the nomenclature}

			The set of solutions of the structure
			equation~\ref{eq:structure_equation} is:

			\begin{equation}
				\setsolstruceq =
					\left\{
						\forcemagvec =
							-\pseudoinv{\strucmat}\wrench_{\platform} +
							\mat{H}\vec{\gain}
						\,\middle|\,
							\vec{\gain \in \Re^{\degofredundancy}}
					\right\}
			\end{equation}

			Where the vectors in $\mat{H}$ form a basis of $\ker{\strucmat}$.

			The set of feasible force distributions is:

			\begin{equation}
				\setoffeasibleforces =
					\left\{
						\forcemagvec \in \Re^{\numcables}
					\,\middle|\,
						0 < \forcemag_{\min} \leq \forcemag_{\indexi} \leq \forcemag_{\max}
							\quad\forall \indexi \in \{1,\ldots,\numcables\}
					\right\}
			\end{equation}

			A pose is therefore wrench feasible if and only if:

			\begin{equation}
				\setsolstruceq \cap \setoffeasibleforces \neq \emptyset
			\end{equation}

			The set $\setsolstruceq \cap \setoffeasibleforces$ is a convex
			polyhedron. That is, if both $\forcemagvec_1$ and $\forcemagvec_2$
			are feasible, than the force distribution

			\begin{equation}
				\forcemagvec = \gain\forcemagvec_1 + (1 - \gain)\forcemagvec_2,
					\quad\forall\gain\in[0;1]
				\label{eq:convexity_of_feasible_forces}
			\end{equation}

			is also a feasible solution

		\subsection{Wrenches Generated by the Robot}%
		\label{sec:wrenches_generated_by_the_robot}

			Similarly to the sets described in
			Section~\ref{sec:wrench_feasible_poses}, the set of wrenches that
			can be generated by the \gls{cdpr} can also be characterised.
			Analogously to Equation~\ref{eq:convexity_of_feasible_forces}, the
			set of wrenches $\setofwrenches$ that the robot can generate at a
			pose $\pose$ is a convex polyhedron. That is, if it is known that a
			robot can generate the discrete set of wrenches:

			\begin{equation}
				\setofwrenches_{\discrete}(\pose) =
					\left\{
						\wrench_1, \ldots, \wrench_{\robotdof}
					\right\}
			\end{equation}

			Then the full set of wrenches the robot is:

			\begin{equation}
				\setofwrenches(\pose) \supseteq \convexhull(\setofwrenches_{\discrete}(\pose))
			\end{equation}

			To ensure that $\setofwrenches$ can be generated, it is sufficient
			to check the $2^{\dim\setofwrenches}$ corners of the set.
			\todo{cite this section explicitly, p 95}

	\subsection{Singularities of \glsentrylongpl{cdpr}}%
	\label{sec:singularities_of_cdprs}

		The singularities of \glspl{cdpr} may be analysed by collecting the loop
		closure equations~\ref{eq:cdpr_loop_closure} of each cable into the
		form:

		\begin{equation}
			\loopclosureconstraints(\pose,\cablelengths) = \zerovec
		\end{equation}

		Differentiation of this expression leads to:

		\begin{equation}
			\frac
			{%
				\partial\loopclosureconstraints(\pose,\cablelengths)
			}
			{%
				\partial\pose
			}
			\delta\pose
			+
			\frac
			{%
				\partial\loopclosureconstraints(\pose,\cablelengths)
			}
			{%
				\partial\cablelengths
			}
			\delta\cablelengths
			= \jac_A\delta\pose + \jac_B\delta\cablelengths
			= \zerovec
		\end{equation}
		\todo{$\jac_A$ and $\jac_B$ are not in nomenclature}

		Where

		\begin{equation}
			\jac_A =
				-\left[
					\begin{matrix}
						2\left(\proximalanchor_1 - \rotmatbare\distalanchor_1 - \transvec\right) &
						\cdots &
						2\left(\proximalanchor_{\robotdof} - \rotmatbare\distalanchor_{\robotdof} - \transvec\right) &
						\\
						%
						2\rotmatbare\distalanchor_1 \times
							\left(
								\proximalanchor_1 - \rotmatbare\distalanchor_1 - \transvec
							\right) &
						\cdots &
						2\rotmatbare\distalanchor_{\robotdof} \times
							\left(
								\proximalanchor_{\robotdof} - \rotmatbare\distalanchor_{\robotdof} - \transvec
							\right)
					\end{matrix}
				\right]
		\end{equation}

		and

		\begin{equation}
			\jac_B =
				\left[
					\begin{matrix}
						2\cablelength_1 & & \\
						& \ddots & \\
						& & 2\cablelength_{\numcables}
					\end{matrix}
				\right]
		\end{equation}

		The robot loses a degree of freedom if
		\(
			\det\jac_B = 0
		\), that is, if a cable is fully retracted. The robot loses its ability
		to withstand certain wrenches if
		\(
			\det\jac_A = 0
		\).

		Since $\jac_A$ is not necessarily square, a necessary and sufficient
		condition to test for its singularity is:

		\begin{equation}
			\rank \strucmat(\pose) < \robotdof
		\end{equation}
		\todo{cite page 145}

%	\subsection{\glsentrylong{dgm}}
%
%	\subsection{\glsentrylong{igm}}
%
%	\subsection{\glsentrylong{dkm}}
%
%	\subsection{\glsentrylong{ikm}}
%
%	\subsection{\glsentrylong{ddm}}
%
%	\subsection{\glsentrylong{idm}}
%
%	\subsection{Workspace Evaluation}
%
%	\subsection{Cable-Cable Collisions}
%
%	\subsection{Cable-Platform Collisions}
