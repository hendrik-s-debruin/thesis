\chapter{Overview}%
\label{chap:overview}

	The goal of this thesis is to develop a path planning and trajectory
	generation architecture for \glspl{cdpr}. A randomised sampling algorithm
	based on the \gls{rrt} algorithm is developed.\todo{ref rrt}

	This thesis makes a distinction between a path $\pathsym$ of the robot
	$\robot$, and its corresponding trajectory $\traj$. In particular, a path is
	defined as the set $\setofposes$ of all poses $\pose$ in configuration space
	that the robot moves through while going from the initial pose
	$\pose_{\initial}$ to its goal pose $\pose_{\goal}$.

	In contrast, a trajectory is concerned with the way the robot visits each
	pose. That is, the trajectory determines at what time the robot must be at a
	certain position along the trajectory.

	The path and trajectory are subject to a range of constraints as summarised
	in symbolic form in Equation~\ref{eq:constraints}.

	\begin{subnumcases}
		{
			\text{find }
			\traj(\timesym), \pathsym(\timenorm) \suchthat
			\label{eq:constraints}
		}
		\pathsym(0)																			&$= {\pose}_{\initial}$																							\label{eq:constraint:start_initial}\\
		\pathsym(1)																			&$= {\pose_{\goal}}$																							\label{eq:constraint:finish_goal}\\
		\traj(\timesym) \mapsto \pathsym(\timenorm)											&$\forall\timesym \forall\timenorm \in [0, 1]$																	\label{eq:constraint:trajectory_maps_to_path}\\
		%\traj(0)																			&$= \pathsym(0)$																								\label{eq:constraint:}\\
		%\traj(\timenorm_{\final})															&$= \pathsym(1)$																								\label{eq:constraint:}\\
		\contdeggeombare(\pathsym)															&$\geq \gain_{\contdeggeombare} \quad \forall \timenorm \in [0, 1]$												\label{eq:constraint:geometric_differentiablity}\\
		\contdegbare(\traj)																	&$\geq \gain_{\contdegbare} \quad\forall\timesym$																\label{eq:constraint:kinematic_differentiability}\\
		\robot(\pathsym(\timenorm)) \cap \obstacle											&$= \emptyset \quad\forall\timenorm \forall\obstacle$															\label{eq:constraint:end_effector_obstacle_collisions}\\
		\cable(\pathsym(\timenorm)) \cap \obstacle											&$= \emptyset \quad\forall\timenorm \forall\obstacle \forall\cable$												\label{eq:constraint:cable_obstacle_collisions}\\
		\robot(\pathsym(\timenorm)) \cap \cable(\pathsym(\timenorm))						&$= \emptyset \quad \forall\timenorm \forall\cable$																\label{eq:constraint:end_effector_cable_collisions}\\
		\cable_{\indexi}(\pathsym(\timenorm)) \cap \cable_{\indexj}(\pathsym(\timenorm))	&$= \emptyset \quad \forall\timenorm \forall\cable_{\indexi} \forall(\cable_{\indexj} \neq \cable_{\indexi})$	\label{eq:constraint:cable_cable_collisions}\\
		\force_{\cable}(\traj)																&$\in [{\force_{\cable}}_{\min}, {\force_{\cable}}_{\max}] \forall\cable \forall\timesym$					\label{eq:constraint:positive_cable_tensions}\\
		%%%%%%%%%% long entry
		\frac{
			\der^n
				\invgeometricmodel
				\left(
					\traj(\timesym)
				\right)
		}
		{
			\der\timesym^n
		}
																							&$\leq \tdern{\cablelengths}{n}_{\max} \quad \forall\timesym$													\label{eq:constraint:kinematic_limits}\\
		%%%%%%%%%% end long entry
		\min\dist_{\pathsym}(\pose_{\initial}, \pose_{\goal})																																				\label{eq:constraint:minimise_distance}\\
		\max{\int}_{\pathsym}\capacitymargin																																						\label{eq:constraint:capacity_margin}
	\end{subnumcases}

	A brief interpretation of the constraints is offered now.

	Constraints~\ref{eq:constraint:start_initial}
	and~\ref{eq:constraint:finish_goal} impose that the path must start at the
	initial pose and finish at the goal pose.
	Constraint~\ref{eq:constraint:trajectory_maps_to_path} requires that the
	function describing the trajectory map to the function describing the path.
	That is, for any $\timenorm \in [0, 1]$, exactly one value for $\timesym$
	can be found such that:

	\begin{equation}
		\traj(\timesym) = \pathsym(\timenorm)
	\end{equation}

	Since \glspl{cdpr} are actuated by cables and not rigid links, they may
	experience vibrations\todo{ref something here}. To mitigate this, an aim of
	this thesis is to generate paths and trajectories that are as smooth as
	possible. Constraint~\ref{eq:constraint:geometric_differentiablity} requires
	that the degree of geometric differentiability of the path be at least
	larger than some tunable value $\gain_{\contdeggeombare}$. For instance, if:

	\begin{equation}
		\gain_{\contdeggeombare} \geq 2
	\end{equation}

	there will be no sharp corners in $\pathsym$. Similarly,
	constraint~\ref{eq:constraint:kinematic_differentiability} requires that the
	trajectory have a degree of differentiability at least equal to a tunable
	value $\gain_{\contdegbare}$. For instance, by requiring that:

	\begin{equation}
		\gain_{\contdegbare} \geq 4
	\end{equation}

	will ensure that the trajectory has a smooth velocity, acceleration and jerk
	profile. Choosing $\gain_{\contdeggeombare}$ and $\gain_{\contdegbare}$
	therefore has a direct effect on minimising the vibrations imposed on the
	robot.

	Constraints~\ref{eq:constraint:end_effector_obstacle_collisions}
	through~\ref{eq:constraint:cable_cable_collisions} require that no
	collisions occur at any point in the path. Due to the nature of
	\glspl{cdpr}, different classes of collisions are possible.
	Constraint~\ref{eq:constraint:end_effector_obstacle_collisions} requires
	that the end effector not collide with any obstacle $\obstacle$.
	Constraint~\ref{eq:constraint:cable_obstacle_collisions} requires that no
	cable $\cable$ collide with any obstacle.
	Constraints~\ref{eq:constraint:end_effector_cable_collisions}
	and~\ref{eq:constraint:cable_cable_collisions} require that the cables not
	collide with the end effector or with each other, respectively.

	Cables can only operate in tension, captured by
	Constraint~\ref{eq:constraint:positive_cable_tensions}.

	Constraint~\ref{eq:constraint:kinematic_limits} requires that the requested
	velocities, accelerations and higher derivatives of the cable position
	adhere to the kinematic limits of the actuators.

	Finally, Constraint~\ref{eq:constraint:minimise_distance}
	and~\ref{eq:constraint:capacity_margin} attempt to minimise the distance
	along the path and maximise the capacity margin along the path. These
	constraints are often in conflict.

	\section{Information Flow in the Architecture}

		A high-level schematic overview of the information flow in the developed
		architecture can be seen in
		Figure~\ref{fig:architecture_information_flow}.

		\begin{figure}[hbt]
			\centering
			\def\svgwidth{\columnwidth}
			\import{res/img/}{architecture_information_flow.pdf_tex}
			\caption{Architecture Information Flow}
			\label{fig:architecture_information_flow}
		\end{figure}\todo{make these colours not shit}

		The architecture is fairly flexible in that it can read the
		configuration of the robot and obstacles from a configuration file. The
		same architecture can therefore be used on multiple \glspl{cdpr} without
		modifying the source code. This configuration is read during the
		initialisation phase.

		Next, the architecture starts looking for a path through sampling. This
		part of the architecture is responsible for finding a path and
		satisfying Constraints~\ref{eq:constraint:start_initial}
		and~\ref{eq:constraint:finish_goal}, as well as the collision
		constraints~\ref{eq:constraint:end_effector_obstacle_collisions}
		through~\ref{eq:constraint:cable_cable_collisions}. Furthermore, it
		assures that cable tensions remain within bounds
		(Constraint~\ref{eq:constraint:positive_cable_tensions}).

		The post-processing step attempts to make the path smooth to adhere to
		Constraint~\ref{eq:constraint:geometric_differentiablity}. It also tries
		to adapt to the path to simultaneously try and satisfy
		Constraints~\ref{eq:constraint:minimise_distance}
		and~\ref{eq:constraint:capacity_margin}.

		Finally, the architecture builds a trajectory along the path
		(Constraint~\ref{eq:constraint:trajectory_maps_to_path}) subject to the
		smoothness Constraint~\ref{eq:constraint:kinematic_differentiability}
		and kinematic Constraints~\ref{eq:constraint:kinematic_limits}.

	\section{Structure of the Thesis}

		The thesis starts with a discussion on how collisions are detected in
		Chapter~\ref{chap:collision_detection}. Chapter~\ref{chap:sampling}
		focusses on the algorithms used to find a collision-free path, while
		Chapter~\ref{chap:path_processing} discusses various post-processing
		steps that are performed to improve the quality of the path.
		Chapter~\ref{chap:motion_law} discusses how a trajectory along the found
		path is generated. An overview of the architecture is given in
		Chapter~\ref{chap:architecture_developed}. Finally,
		Chapter~\ref{chap:experimental_results} discusses some experimental
		results.
