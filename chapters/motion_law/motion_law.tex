\chapter{Motion Law}%
\label{chap:motion_law}

	The path found with the algorithms described in
	Chapter~\ref{chap:path_processing} can be considered as a continuous smooth
	mapping that adheres to:

	\begin{equation}
		\pathsym : \timenorm \in [0, 1] \mapsto \specialEuclideanGroup{3}
			\quad
			\suchthat
			\quad
			\pathsym(0) = \pose_{\initial}, \pathsym(1) = \pose_{\goal}
	\end{equation}

	If $\timenorm$ is interpreted as time, then it would imply that the robot is
	in its initial pose at $\timesym = 0\si{\second}$ and reaches its goal pose
	at $\timesym = 1\si{\second}$. While $\pathsym$ is guaranteed to be free of
	collisions, this interpretation makes no guarantee that the actuators can
	move the end-effector this fast. For this reason, a motion law,
	$\motionlaw$, must be found that performs the following mapping:

	\begin{equation}
		\motionlaw: \timesym\in\Re \mapsto \timenorm\in[0, 1]
	\end{equation}

	The trajectory, $\traj:\timesym \mapsto \specialEuclideanGroup{3}$, can then
	be defined simply as:

	\begin{equation}
		\traj = \pathsym(\motionlaw(\timesym))
	\end{equation}

	In addition to the no-collision guarantees of $\pathsym$, $\traj$ also
	guarantees that the actuators stay within their saturation ranges. The
	present chapter describes the approach followed to find a suitable
	$\motionlaw$.

	\section{Motion Law Generation Algorithm}%
	\label{sec:motion_law_generation_algorithm}

		Since the actuators work directly on the length of the cables, the
		search for a suitable motion law $\motionlaw$ is performed in cable
		space and not in pose space. As a first step, $\timenorm$ is considered
		to be linked directly to time, $\timesym \defeq \timenorm$, and the
		cable velocities along the resulting trajectory is then found as:

		\begin{equation}
			\dot{\cablevec}(\timenorm) =
				\invgeometricmodel
				\left(
					\frac
					{%
						\der\pathsym
					}
					{%
						\der\timesym
					}
				\right)
		\end{equation}

		Each cable in $\cablevec$ is subjected to kinematic constraints on its
		velocity, $\dot\cablelength_{\max}$\todo{also discuss acceleration, jerk
		etc when implemented}.  That is, each cable has a maximum velocity which
		it may not exceed. As such, the next step in the algorithm is finding
		the ranges that exceed this maximum velocity. In other words, find the
		set of time instants $\timenorm$ where:

		\begin{equation}
			\abs{\dot{\cablevec}(\timenorm)} \geq \dot{\cablevec}_{\max}
			\label{eq:excessive_cable_velocity}
		\end{equation}

		In general, there may be multiple ranges along the trajectory where the
		maximum velocity in Equation~\ref{eq:excessive_cable_velocity} is
		exceeded. A range in $\timenorm$, $\rangetimenorm$, is encapsulated as
		the ordered tuple:

		%\begin{equation}
		%	\rangetimenorm =
		%		\left(
		%			\timenorm_{\min},
		%			\timenorm_{\max}
		%		\right)
		%	%
		%\end{equation}

		%such that

		%\begin{equation}
		%	\abs
		%	{%
		%		\dot{\cablevec}
		%			(
		%				\timenorm
		%			)
		%	}
		%	%
		%	\geq \dot{\cablevec}_{\max}
		%	\quad
		%	\forall\timenorm\in[\timenorm_{\min}, \timenorm_{\max}]
		%\end{equation}

		%and

		%\begin{align}
		%	\begin{split}
		%		\dot{\cablevec}(\timenorm_{\min}^-) &< \dot{\cablevec}_{\max} \\
		%		\dot{\cablevec}(\timenorm_{\min}^+) &> \dot{\cablevec}_{\max} \\
		%		\dot{\cablevec}(\timenorm_{\max}^-) &> \dot{\cablevec}_{\max} \\
		%		\dot{\cablevec}(\timenorm_{\max}^+) &< \dot{\cablevec}_{\max} \\
		%	\end{split}
		%\end{align}

		\begin{equation}
			\rangetimenorm =
				\left(
					\timenorm_{\min},
					\timenorm_{\max}
				\right)
			%
			\quad
			\suchthat
			\quad
			%
			\begin{cases}
				\abs
				{%
					\dot{\cablevec}
						(
							\timenorm
						)
				}
				\geq \dot{\cablevec}_{\max}
				\forall\timenorm\in[\timenorm_{\min}, \timenorm_{\max}]\\
				%
				\dot{\cablevec}(\timenorm_{\min}^-) < \dot{\cablevec}_{\max} \\
				\dot{\cablevec}(\timenorm_{\min}^+) > \dot{\cablevec}_{\max} \\
				\dot{\cablevec}(\timenorm_{\max}^-) > \dot{\cablevec}_{\max} \\
				\dot{\cablevec}(\timenorm_{\max}^+) < \dot{\cablevec}_{\max} \\
			\end{cases}
		\end{equation}

		Note that the following notation is used to test whether a value is in a
		range:

		\begin{equation}
			\timenorm \in \rangetimenorm =
				\begin{cases}
					\true, \quad \timenorm \in [\timenorm_{\min},
					\timenorm_{\max}] \\
					\false, \quad\text{otherwise}
				\end{cases}
		\end{equation}

		All such ranges are put into an ordered set of ranges,
		$\setofrangestimenorm$, such that:

		\begin{equation}
			{\rangetimenorm}_{\indexi} \prec
			{\rangetimenorm}_{\indexj} \iff
			{\rangetimenorm}_{\indexi}\code{.}{\timenorm_{\max}} <
			{\rangetimenorm}_{\indexj}\code{.}{\timenorm_{\min}}
		\end{equation}

		$\setofrangestimenorm$ is then used as an input to the algorithms that
		generate $\motionlaw$.

		The idea behind the algorithm is now to simply scale the time in such a
		way that:

		\begin{equation}
			\frac
			{%
				\der\timesym
			}
			{%
				\der\timenorm
			}
			=
			\begin{cases}
				1, \quad \timenorm \notin \setofrangestimenorm \\
				\gain_{\indexi}, \quad \timenorm \in {\rangetimenorm}_{\indexi}
			\end{cases}
			\forall{\rangetimenorm}_{\indexi}\in\setofrangestimenorm
		\end{equation}

		Where $\gain_{\indexi}$ is chosen such that, for the currently active
		${\rangetimenorm}_{\indexi}$:

		\begin{equation}
			\abs
			{%
				\dot{\cablevec}(\timenorm)
			}
			\leq \dot{\cablevec}_{\max}
			\quad\forall{\timenorm}\in{\rangetimenorm}_{\indexi}
		\end{equation}
		\todo{describe how $\gain_{\indexi}$ is chosen}

		A graphical intuition of the algorithm is given in
		Figure~\ref{fig:motion_law_graphical_intuition}. The top graph shows the
		velocity profile of a single cable for a given path
		$\pathsym(\timenorm)$. It also highlights the set $\setofrangestimenorm$
		that exceed the maximum velocities. The middle graph shows schematically
		how the motion law $\motionlaw(\timesym) \mapsto \timenorm$ is
		generated. Finally, the bottom graph shows the effect of applying
		$\motionlaw$ to $\pathsym$ to obtain a trajectory $\traj(\timesym)$
		which no longer exceeds maximum velocities.

		\begin{figure}[hb]
			\centering
			\def\svgheight{8cm}
			\import{res/img/}{motion_law_intuition.pdf_tex}
			\caption{Motion Law Construction}%
			\label{fig:motion_law_graphical_intuition}
		\end{figure}


