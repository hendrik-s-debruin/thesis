\section{Conclusion and Future Work}

	This paper documented a path-planning method for \glspl{cdpr}. The algorithm
	is based on \gls{rrt}. Methods to detect different classes of collisions
	were discussed an implemented. The method takes into account the static
	quality of poses along the path by using the capacity margin. B-splines are
	used to generate a final smooth path. A simple motion-law generation
	strategy is also discussed. The motion law makes use of B-splines to ensure
	continuity in the velocity and acceleration profiles of the actuators.

	This paper made the assumption that the dynamic effects of the capacity
	margin are negligible for the planning stage. The conditions under which
	this assumption is valid still need to be investigated.

	This paper discusses many post-processing steps. While they each achieve
	their respective goals, running all of them can add considerable time to the
	overall planning process. Methods of tuning them to trade calculation speed
	for post-processing quality must still be investigated.
