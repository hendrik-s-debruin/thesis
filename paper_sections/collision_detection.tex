\section{Collision Detection}

	This section gives an overview of the collision detection methods used in
	the planning algorithm.

	\subsection{Cable-Cable Collision}

		Due to Assumption~\ref{assumption:cable_slack}, the cables may be
		modelled as straight lines. The planning algorithm considers cables to
		be in collision if any point on the cable is within a distance $\tol$ of
		some point on another cable. Consider
		Figure~\ref{fig:cable_collision_detection}.

		\begin{figure}[hb]
			\centering
			\def\svgwidth{\columnwidth}
			\import{res/img/}{cable_collision_detection.pdf_tex}
			\caption{Cable-Cable Collision Detection}
			\label{fig:cable_cable_collision_detection}
		\end{figure}

		The two cables, $\cable_1$ and $\cable_2$, may be written in parametric
		form as:

		\begin{equation}
			\cable_i = \distalanchor_i + \timenorm_i(\proximalanchor_i -
			\distalanchor_i)
		\end{equation}

		Where $\timenorm_i \in [0, 1]$. Finding the point where these two cables
		meet, $\cable_1 = \cable_2$:

		\begin{align}
			\left[
				\begin{matrix}
					\proximalanchor_1 - \distalanchor_1 &
					\distalanchor_2 - \proximalanchor_2
				\end{matrix}
			\right]
			\left[
				\begin{matrix}
					\timenorm_1 \\
					\timenorm_2 \\
				\end{matrix}
			\right]
			&=
			\left[
				\begin{matrix}
					\distalanchor_2 - \distalanchor_1
				\end{matrix}
			\right]
			%
			\\
			%
			\mat{A}\vec{\timenorm} &= \vec{b}%
			\label{eq:cable_cable_interference}
		\end{align}

		For the spatial case, this equation can be solved for the least distance
		between the two cables by using the pseudo-inverse:

		\begin{equation}
			\vec{\timenorm} = \pseudoinv{\mat{A}}\vec{b}
		\end{equation}

		However, this finds the shortest distance between the infinite lines
		along the cables. To find the shortest distance between the finite
		cables, the solution needs to be clamped as follows:

		\begin{equation}
			\timenorm_i =
			\begin{cases}
				0, \quad \timenorm_i < 0 \\
				1, \quad \timenorm_i > 1 \\
				\timenorm_i, \quad \text{otherwise}
			\end{cases}
		\end{equation}

		The two cables are then in collision if:

		\begin{equation}
			\norm{\cable_1(\timenorm_1) - \cable_2(\timenorm_2)} < \tol
		\end{equation}

	\subsection{Cable-Obstacle Collisions}

		A cable is in collision with an obstacle if it intersects any of the
		obstacle's faces. As such, it is sufficient to present an algorithm that
		detects whether a cable $\cable$ intersects a single face $\face$ of an
		obstacle $\obstacle$ to detect such collisions.

		Consider Figure~\ref{fig:cable_obstacle_collision}. Due to
		Assumption~\ref{assumption:convex_polyhedra}, each face of the obstacles
		is modelled as a convex polygon in 3D space.

		\begin{figure}[hbt]
			\centering
			\def\svgwidth{\columnwidth}
			\import{res/img/}{cable_obstacle_collision.pdf_tex}
			\caption{Cable-Obstacle Collision Detection}%
			\label{fig:cable_obstacle_collision_detection}
		\end{figure}

		A plane $\plane$ through the face $\face$ can be written in parametric
		form as:

		\begin{equation}
			\plane = \point_1 + \timenorm_1(\point_2 - \point_1) +
			\timenorm_2(\point_3 - \point_1)
		\end{equation}

		The algorithm first determines whether the finite cable intersects the
		infinite extension of the plane $\plane$. By assuming the cable is of
		infinite length, the point $\cable = \plane$ is found by:

		\begin{align}
			\left[
				\begin{matrix}
					\proximalanchor - \distalanchor &
					\vertex_1 - \vertex_2 &
					\vertex_1 - \vertex_2
				\end{matrix}
			\right]
			\left[
				\begin{matrix}
					\timenorm_1 \\
					\timenorm_2 \\
					\timenorm_3
				\end{matrix}
			\right]
			&=
			\left[
				\begin{matrix}
					\vertex_1 - \distalanchor
				\end{matrix}
			\right]
			%%
			\\
			%%
			\mat{A}\vec{\timenorm} &= \vec{b}
			%%
			\\
			%%
			\vec{\timenorm} &= {\mat{A}}^{-1}\vec{b}
		\end{align}

		Now, the finite cable intersects the infinite plane if:

		\begin{equation}
			\timenorm_1 \in [0, 1]
		\end{equation}

		If this is not the case, the algorithm may exit early. Next, it is
		necessary to determine whether the cable intersects the face of the
		object. Note, however, that in general it is not sufficient to simply
		test that $\timenorm_1, \timenorm_2 \in [0, 1]$. Instead, the convex
		hull of the vertices in $\face$ is used to determine whether the point
		intersects the face, $\cable(\timenorm_1) \in \face$.
		\todo{cite convex hull or something}

		The full collision criterion becomes:

		\begin{equation}
			\begin{cases}
				\timenorm_1 &\in [0, 1]\\
				\cable(\timenorm_1) &\in \face_i
			\end{cases}
			\label{eq:cable_obstacle_collision_condition}
		\end{equation}

	\subsection{Cable-End-Effector Collisions}

		Since the end-effector and obstacles are modelled the same way, the same
		algorithm can be used to detect collisions between cables and the
		end-effector or obstacles. However, since the cables are fixed to the
		end-effector at the anchor, the anchor point needs to be removed from
		consideration for collisions. The collision criterion for
		cable-end-effector collisions is thus:

		\begin{equation}
			\begin{cases}
				\timenorm_1 &\in [0, 1)\\
				\cable(\timenorm_1) &\in \face_i
			\end{cases}
			\label{eq:cable_robot_collision_condition}
		\end{equation}

	\subsection{End-Effector-Obstacle Collisions}

		Many algorithms are available for collision detection between convex
		polyhedra. This paper makes use of the well-known Separating Axis
		Theorem. \todo{cite}

	\subsection{Capacity Margin}\todo{cite capacity margin}

		The planning architecture makes use of the capacity margin
		$\capacitymargin$ to evaluate the validity of poses along the path.
		While not a collision in the traditional sense, a pose with a negative
		capacity margin is not valid. Therefore, a pose is considered in
		collision for planning purposes if, for some threshold $\tol$:

		\begin{equation}
			\capacitymargin < \tol
		\end{equation}
