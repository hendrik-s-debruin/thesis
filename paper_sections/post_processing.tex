\section{Post Processing}

	This section documents the path processing steps followed in this paper. The
	algorithm developed first tries to simplify the path by removing unnecessary
	intermediate poses and optimising the rotation trajectory. Next, it tries to
	locally optimise the capacity margin of the remaining poses and finally
	smoothly interpolates the path using B-splines.

	\subsection{Pose Removal}

		The path planning returns an ordered set of poses, $\setofposes$, such
		that, by connecting each pose with a straight line, the goal pose may be
		reached from the start pose without collisions. Due to the random nature
		of \gls{rrt}, however, there may be many unnecessary poses in
		$\setofposes$.

		Two simple algorithms are employed to remove unnecessary poses and
		shorten the total distance as measured by
		Equation~\ref{eq:distance_measure_capcity_margin}.

		\subsubsection{Reducing Poses in $\setofposes$}

			The first algorithm simply removes a pose $\pose_i$ from
			$\setofposes$ if:

			\begin{itemize}

				\item

					Doing so decreases the total distance:

					\begin{equation}
						\dist(\pose_{i - 1}, \pose_{i + 1}) <
							\dist(\pose_{i - 1}, \pose_i)
							+ \dist(\pose_i, \pose_{i + 1})
					\end{equation}

				\item

					The straight-line path between $\pose_{i - 1}$ and $\pose_{i
					+ 1}$ is collision-free.

			\end{itemize}

			This algorithm is run recursively until no more poses can be
			removed.

		\subsubsection{Removing Corners from $\setofposes$}

