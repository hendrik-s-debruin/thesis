\section{Planning}

	This section gives an overview of the planning step of the algorithm.

	\subsection{Sampling}

		The approach followed is based on the \gls{rrt} \todo{cite} algorithm.
		For this paper, a pose $\pose$ is defined as a tuple of the
		end-effector's translation $\transvec$ and orientation quaternion
		$\quaternion$:

		\begin{equation}
			\pose = (\transvec, \quaternion)
		\end{equation}

		The end-effector may start or stop in any pose. However, during the
		trajectory the end-effector is kept upright. This is done for the
		following reasons:

		\begin{enumerate}

			\item

				By only searching for upright poses, the search space loses two
				degrees of freedom and is therefore greatly reduced.

			\item

				By rotating around only one axis for the majority of the path,
				complicated and potentially unnecessary rotations around
				multiple axes are avoided.

			\item

				For pick-and-place operations, it may be desirable to keep the
				end-effector steady and upright through the trajectory.

		\end{enumerate}

		Note that while the end-effector is kept upright in the middle of the
		trajectory, it is still allowed to do complex rotations at the start and
		end of the path in order to reach arbitrary start and stop poses. This
		is shown schematically in Figure~\ref{fig:pose_sampling}.

		\begin{figure}[hb]
			\centering
			\def\svgwidth{\columnwidth}
			\import{res/img/}{pose_sampling.pdf_tex}
			\caption{Pose Sampling}%
			\label{fig:pose_sampling}
		\end{figure}

		Poses are selected according to the following strategy:

		\begin{enumerate}

			\item

				A translation vector $\transvec$ is drawn from a normal
				distribution such that:
				\label{item:sample_strategy:translation}

				\begin{enumerate}

					\item

						The median of the distribution is the centre of the
						workspace.

					\item

						The edges of the workspace are three standard deviations
						away from the median.
				\end{enumerate}

			\item

				An angle $\theta$ is drawn from a normal distribution with
				median $\pi/2$ and standard deviation $\pi/6$.

				\label{item:sample_strategy:angle}

			\item

				A quaternion is built from the formula:

				\begin{equation}
					\quaternion = 0\vec{i} + \sin(\theta/2)\vec{j} + 0\vec{k} +
						\cos(\theta/2)
				\end{equation}

				\label{item:sample_strategy:quaternion}

				%This produces a rotation of angle $\theta$ around the $y$
				%axis.\todo{explain that y axis is pointing up}

			\item

				The pose $\pose = (\transvec, \quaternion)$ is returned.

			\item

				The pose is moved in the direction of increasing capacity margin
				by applying the following expression $n$ times:

				\label{item:improve_capacity_margin}

				\begin{equation}
					\pose \gets \pose +
						\gain\frac
						{%
							\nabla\capacitymargin
						}
						{%
							\norm
							{%
								\nabla\capacitymargin
							}
						}
					\label{eq:improve_sampled_capacity_margin}
				\end{equation}


		\end{enumerate}

	\subsection{Sampling Justification}

		Translations are chosen according to
		Item~\ref{item:sample_strategy:translation} to create a bias towards the
		center of the workspace. This is done because the capacity margin tends
		to be higher towards the center of the workspace. However, by letting
		the edge of the workspace be within three standard deviations of the
		center, a path can still be found when the center is obstructed by
		obstacles.

		Item~\ref{item:sample_strategy:angle} ensures that most of the sampled
		rotations lie in the range $\theta \in [0, \pi]$, but biased towards
		$\theta = \pi/2$. The rotations are biased in order to try and limit
		unnecessary wide rotations between different sampled poses during the
		planning stage.

		Item~\ref{item:sample_strategy:quaternion} builds a rotation around the
		vertical axis using the angle obtained from
		Item~\ref{item:sample_strategy:angle}.

		Item~\ref{item:improve_capacity_margin} tries to improve the capacity
		margin at the sampled poses. This is done at this step, since taking
		samples is an inexpensive operation. It is therefore better to try and
		start from a path with high capacity margin poses, as improving the
		capacity margin with post-processing operations can become expensive.
		This is because sampling is concerned only with the current pose,
		whereas post-processing needs to guarantee all the poses along the
		entire path.

	\subsection{Path Building}

		As in other \gls{rrt} algorithms, a path is built by combining different
		poses to their nearest neighbours provided that there is no collision
		in the path between the poses. This paper simply connects sampled poses
		by straight lines. The path between poses is checked by simply
		discretising the line and checking sufficient points along its course.

	\subsection{Distance in $\specialEuclideanGroup{3}$}

		There is a degree of arbitrariness in how to measure distances in
		$\specialEuclideanGroup{3}$. This paper proposes to use the following
		distance measure in the context of \glspl{cdpr}:

		\begin{equation}
			\dist(\pose_1, \pose_2) =
				\norm{%
					\invgeometricmodel(\pose_2) - \invgeometricmodel(\pose_1)
				}
			\label{eq:distance_measure}
		\end{equation}

		Here, $\invgeometricmodel$ denotes the inverse geometric model of the
		\gls{cdpr} and returns the vector of cable lengths.
		Equation~\ref{eq:distance_measure} measures distance in cable space and
		is measured in meters. This has the following advantages:

		\begin{enumerate}

			\item

				Both orientation and translation are measured using a single
				dimensionally homogeneous measure. There is no need to measure
				translation and orientation separately and no arbitrariness in
				how they are compared.

			\item

				The change in the length of the cables relates directly to how
				much the actuators have to act on the cables. Therefore, this
				distance measure directly encodes the amount of actuation
				required to move from one pose to another.

		\end{enumerate}

	\subsection{Distance Measure used for Planning}

		The planning algorithm does not only try to minimise the distance
		travelled along the entire path, but also attempts to optimise the
		capacity margin along this path. These two objectives are often in
		conflict, since a longer path might have a higher capacity margin. As
		such, during the planning stage, nearest neighbours are chosen according
		to the following distance function:

		\begin{equation}
			\begin{split}
				\dist(\pose_1, \pose_2) =
					&-\gain_{\capacitymargin}
					\frac%
					{%
						\capacitymargin(\pose_2) - \capacitymargin(\pose_1)
					}
					{%
						\forcemag_{\max}
					}\\
					&+
					(1 - \gain_{\capacitymargin})
					\frac%
					{%
						\norm{%
							\invgeometricmodel(\pose_2) - \invgeometricmodel(\pose_1)
						}
					}
					{%
						\diag(\workspace)
					}
					\label{eq:distance_measure_capcity_margin}
			\end{split}
		\end{equation}

		Here, the first term measures the increase in capacity margin from pose
		$\pose_1$ to $\pose_2$. Since the capacity margin should be maximised,
		this term is negative. The second term measures the distance in
		$\specialEuclideanGroup{3}$ as described above.

		The term $\force_{\max}$ is the maximum allowable cable tension, while
		$\diag{\workspace}$ is the diagonal corner-to-corner distance of the
		workspace. These two terms normalise their respective numerators and
		render them dimensionless. Finally, $\lambda_{\capacitymargin} \in [0,
		1]$ is a scaling factor that encodes the relative importance of the
		conflicting objective functions
